
In the Steiner Multicycle Problem (SMCP), we have a graph \(G\), a cost function $c \colon E(G) \to \mathbb{R}_\ge$ and a collection of terminal sets \(\{T_1, \dots , T_k\}\), where for each \(a\) in \([k]\) \(T_a\) is a subset of \(V(G)\) and these terminal sets are pairwise disjoint. The problem consists of finding a minimum cost set of vertex-disjoint cycles such that, for each \(a\) in \([k]\), all vertices of \(T_a\) belong to the same cycle.
There are applications related to routing problems that can be translated into SMCP instances.
More precisely, SMCP models a less-than-truckload scenario, where multiple companies must periodically visit pickup and delivery locations.
To reduce their individual costs, these companies can collaborate to create shared routes that visit multiple pickup and delivery locations to reduce total transportation costs.
Clearly, SMCP is a generalization of the Traveling Salesman Problem and thus does not admit a polynomial-time approximation scheme (PTAS), even in the metric case.
On the positive side, the literature for this problem describes a PTAS for Euclidean instances. In this work, we generalize this result by showing a PTAS for SMCP in graphs of bounded treewidth and, from this, we derive a PTAS for graphs of bounded \textit{genus}.
The proposed algorithm is essentially an adaptation of the PTAS for the Steiner Forest Problem on graphs of bounded \textit{genus}.
Furthermore, we implemented the 3-approximate algorithm for SMCP proposed by \cite{smcp_3apx} on complete metric graphs and performed computational experiments.
The solution costs produced by this algorithm are on average \(34\%\) worse than the corresponding optimal solutions, which is considerably better than the theoretical guarantee.
On the other hand, the running time of the implementation is generally worse than other heuristics presented in the literature.
\hfill\\

\noindent\textbf{Keywords:} Approximation algorithm, graph theory, bounded genus, Steiner Multicycle, computational experiments
