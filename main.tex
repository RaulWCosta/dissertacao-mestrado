% INFORMAÇÃO SOBRE A VERSÃO DESSE DOCUMENTO
% VERSÃO DA CLASSE ppgccufmg 1.60 (de 2021-03-29) - http://vilarneto.com/ppgccufmg (ou https://web.archive.org/web/20210626223024/https://vilarneto.com/ppgccufmg/)

\documentclass[msc]{ppgccufmg}    % ou [msc] para dissertações
                                  % de mestrado. Para propostas ou
                                  % projetos, usar [phd,project],
                                  % [msc,proposal], etc.

% \usepackage[brazil]{babel}        % se o documento for em português, OU
\usepackage[english]{babel}      % se o documento for em inglês
% \usepackage[latin1]{inputenc}
\usepackage[utf8]{inputenc}
\usepackage[T1]{fontenc}
\usepackage{type1ec}
\usepackage{amsthm}
\usepackage{graphicx}
\usepackage{pgfplots}
\usepackage[a4paper,
  portuguese,
  bookmarks=true,
  bookmarksnumbered=true,
  linktocpage,
  colorlinks,
  citecolor=black,
  urlcolor=black,
  linkcolor=black,
  filecolor=black,
  ]{hyperref}
\usepackage[square]{natbib}

\usepackage{tikz, tikz-network}

\usepackage{amsmath, amssymb, amsfonts, amsbsy, algorithmicx, algorithm}

\usepackage[noend]{algpseudocode}

\usepackage{mdframed}
\usepackage{lipsum}

\makeatletter
% \def\BState{\State\hskip-\ALG@thistlm}
\makeatother

% \newtheorem{theorem}{Theorem}
% \newtheorem{theorem}{Teorema}
% \newtheorem{lemma}{Lema}

\newtheorem{theorem}{Theorem}[chapter]
% \newtheorem{lemma}{Lemma}
\newtheorem{lemma}[theorem]{Lemma}
\newtheorem{claim}[theorem]{Claim}
\newtheorem{corollary}[theorem]{Corollary}

%%%% operadores matemáticos
\DeclareMathOperator{\opt}{\textrm{OPT}}
\DeclareMathOperator{\val}{\textrm{val}}
\DeclareMathOperator{\sol}{\textrm{Sol}}

%%% Definições
% \newcommand{\steinercycle}{\textsc{Steiner Multicycle}}

% definicoes
\newcommand{\steinercycle}{\textsc{SMCP}}
\newcommand{\steinercyclerestricted}{\textsc{R-SMCP}}
\newcommand{\steinerforest}{\textsc{SFP}}
\newcommand{\ptas}{\textsc{PTAS}}
\newcommand{\nonpoly}{\mathcal{NP}}
\newcommand{\nonpolyopt}{\mathcal{NPO}}

\newcommand{\poly}{\mathcal{P}}

\usepackage{algpseudocode}

% muda nomes de comandos do algorithmicx 
\algrenewcommand\algorithmicrequire{\textbf{Input:}}
\algrenewcommand\algorithmicensure{\textbf{Output:}}

\newenvironment{ftheo}
  {\begin{mdframed}\begin{theorem}}
  {\end{theorem}\end{mdframed}}

\newenvironment{flemma}
  {\begin{mdframed}\begin{lemma}}
  {\end{lemma}\end{mdframed}}

\newenvironment{fcorollary}
  {\begin{mdframed}\begin{corollary}}
  {\end{corollary}\end{mdframed}}

\begin{document}

% O comando a seguir, \ppgccufmg, provê todas as informações relevantes para a
% classe ppgccufmg. Por favor, consulte a documentação para a descrição de
% cada chave.

% Um exemplo para documentos em português é apresentado a seguir:
\ppgccufmg{
  title={An algorithmic study of the Steiner Multicycle Problem},
  authorrev={Martins Costa, Raul Wagner},
  %cutter={D1234p},
  %cdu={519.6*82.10},
  university={Federal University of Minas Gerais},
  course={Computer Science},
  % portuguesetitle={Um estudo algorítmico do Problema de Multiciclos de Steiner},
  portugueseuniversity={Universidade Federal de Minas Gerais},
  portuguesecourse={Ciência da Computação},
  address={Belo Horizonte},
  date={2024-02},
  % keywords={Algoritmos de Aproximação,Teoria de Grafos,Otimização,Complexidade Computacional},
  % postkeywords={Algoritmos de Aproximação,Teoria de Grafos,Otimização,Complexidade Computacional},
  % indexkeys={},
  advisor=[male]{Prof. Dr. Phablo F. S. Moura},
%   coadvisor=[female]{Maria Teresa Eleonora},
%  approval={img/approvalsheet.eps},
%  approval=[-2.5cm][1]{aprovalsheet},
  abstract={Resumo}{capitulos/abstract_port},
  abstract={Abstract}{capitulos/abstract},
%   abstract={Resumo Estendido}{capitulos/resumoest},
  % dedication={capitulos/dedicatoria},
  ack={capitulos/agradecimentos},
%  ack=[Acknowledgments]{ack},
%   epigraphtext={A verdade é o contrário da mentira, \\
%     e a mentira é o oposto da verdade.}{Autor desconhecido},
  indexkeys={1.~Computação --- Teses. 2.~Algoritmos --- Teses. I.~Orientador.
    II.~Título.},
  beforetoc={\listofalgorithms}
}

% \keywords{}
% \keywords{Algoritmos de Aproximação, Teoria de Grafos, Otimização, e Complexidade Computacional}


% Um exemplo para documentos em inglês é apresentado a seguir (lembre-se de
% usar \usepackage[english]{babel}):
%\ppgccufmg{
%  title={Protocol for Error-Verification inside\\Totally Error-Free
%    Networks},
%  authorrev={da Camara Neto, Vilar Fiuza},
%  cutter={D1234p},
%  cdu={519.6*82.10},
%  university={Federal University of Minas Gerais},
%  course={Computer Science},
%  portuguesetitle={Protocolo para Verificação de Erros\\em Redes Totalmente
%    Confiáveis},
%  portugueseuniversity={Universidade Federal de Minas Gerais},
%  portuguesecourse={Ciência da Computação},
%  address={Belo Horizonte},
%  date={2008-03},
%  advisor={Adamastor Pompeu Setúbal},
%  approval={img/approvalsheet.eps},
%  abstract=[brazil]{Resumo}{resumo},
%  abstract={Abstract}{abstract},
%  dedication={dedicatoria},
%  ack={agradecimentos},
%  epigraphtext={Truth and lie are opposite things.}{Unknown},
%}




% \newcommand{\dummytxt}{\dummytxta\dummytxtb\dummytxtc}

\chapter{Introduction}
\label{chapter:introduction}

The Steiner Multicycle Problem (\(\steinercycle\)) was introduced by \cite{Pereira2018TheSM} in the context of vehicle routing where multiple companies must visit pickup and delivery locations. That way, a group of companies can collaborate to realize the necessary deliveries and reduce the total cost of transportation. 
Notably, one assumes that the companies must make the deliveries periodically, and the size of each delivery is much smaller than the vehicles' capacities.
The interested reader may find more details about this application in \cite{Pereira2018TheSM}.

The problem \(\steinercycle\) has as input an undirected graph \(G\), a function $c \colon E(G) \to \mathbb{R}_\ge$ that associates a cost to each edge of \(G\), and a set \(\{T_1, \dots, T_k\}\) of pairwise disjoint subsets of vertices of \(G\) (a.k.a. \textbf{terminal sets}). The goal of the problem is to find a minimum-cost set of cycles \(\mathcal{C}\), such that each terminal set \(T_i\) is contained within the same cycle.

A solution to the Steiner Multicycle Problem can be represented by a single multiset that contains all the edges forming the cycles in the solution. Moreover, since a multiset of edges and the cycles induced by these edges are equivalent, throughout this work, we refer to the solution as both the multiset of edges and the subgraph of \(G\) induced by those edges interchangeably.

The solution cost is defined as the sum of all associated costs of the edges in the solution multiset.

The problem is defined next. An instance of \(\steinercycle\) and a feasible solution are depicted in Figure~\ref{fig:exem_multicycle}.

\medskip
\noindent \fbox{
	\parbox{.97\textwidth}{
		\noindent
		\textsc{Steiner Multicycle Problem (\textbf{SMCP})}\\
		\noindent
		\textbf{Instance}: A graph \(G\), $c \colon E(G) \to \mathbb{R}_\ge$, and a set \(\{T_1, \dots, T_k\}\) of pairwise disjoint subset of vertices of \(G\) (i.e. the terminals).\\
		%\noindent
		%\textbf{Parameter}: $k$.
		\noindent
		\textbf{Goal}: Find in \(G\) a minimum cost set of cycles such that all terminals in \(T_i\) belong to the same cycle, for each \(i \in [k]\).
	}
}
\medskip

\begin{figure}[H]
    \centering
\begin{tikzpicture}

\Vertex[x=0, y=2, label=$t'_3$, color=white]{A}
\Vertex[label=$t'_2$, color=white]{B}
\Vertex[x=2, y=2, color=white]{C}
\Vertex[x=2, y=0, color=white]{D}
\Vertex[x=2, y=4, label=$t_2$, color=white]{E}
\Vertex[x=4, y=2, label=$t_3$, color=white]{F}
\Vertex[x=4, y=0, color=white]{G}
\Vertex[x=6, y=2, label=$t_1$, color=white]{H}
\Vertex[x=6, y=0, label=$t'_1$, color=white]{I}

\Edge[lw=3pt, label=$e_1$, position=left](A)(B)
\Edge[lw=4pt, label=$e_2$, position=above](A)(C)
\Edge[lw=4pt, label=$e_3$, position=above](B)(D)
\Edge[lw=4pt, label=$e_4$, position=left](C)(D)
\Edge[lw=4pt, label=$e_5$, position=left](C)(E)
\Edge[lw=4pt, label=$e_6$, position=above](C)(F)
\Edge[label=$e_7$, position=above left](D)(F)
\Edge[label=$e_8$, position=above](D)(G)
\Edge[lw=4pt, label=$e_9$, position=above right](E)(F)
\Edge[lw=4pt, label=$e_{10}$, position=above](G)(I)
\Edge[lw=4pt, label=$e_{11}$, position=above left](G)(H)

\end{tikzpicture}
    \caption{An example of a feasible solution for the \(\steinercycle\) considering the set of pairs of terminals \(\{\{t_1, t'_1\}, \{t_2, t'_2\}, \{t_3, t' _3\}\}\). The solution is represented by the edges in bold and corresponds to the multiset \(\{e_1, e_2, e_3, e_4, e_5, e_6, e_9, e_{10}, e_{11}, e_{10}, e_{11}\}\). The weights of the edges were omitted for simplicity.}
    \label{fig:exem_multicycle}
\end{figure}

\cite{Pereira2018TheSM} studied the problem for complete metric graphs (i.e., complete graphs that guarantee the triangular inequality for the edges costs). The authors presented a 4-approximation algorithm, a heuristic algorithm, and a mixed-integer linear formulation for the problem.
Moreover, the authors performed computational experiments comparing the proposed heuristic, called \textit{Refinement Search}, to a GRASP implementation. In summary, the results showed that the proposed heuristic reached better quality results in less time.

\citeauthor{Pereira2018TheSM} also introduced a restricted version of the Steiner Multicycle Problem (R-\(\steinercycle\)) in which every terminal set has only two vertices, one for pickup and the other for delivery. \cite{LINTZMAYER2020134} showed an equivalence between the two problems -- \(\steinercycle\) and R-\(\steinercycle\) -- by proposing a simple transformation from instances of the \(\steinercycle\) to R-\(\steinercycle\), which we present below. 

Let \(\mathcal{T} = \{t_1, \dots, t_\ell\}\) be a set of terminals, and create \(\ell\) new vertices \(\{u_1, \dots, u_\ell\}\). For each \(i \in \{1, \ldots, \ell\}\), we create an edge \(\{t_i, u_i\}\) of cost zero. We define \(\{\{u_1, t_2\}, \dots, \{u_{\ell-1}, t_\ell\}, \{u_\ell, t_1\}\}\) as terminals pairs of an instance of R-\(\steinercycle\). Note that the set of edges that forms a solution of \(\steinercycle\) with terminals of \(\mathcal{T}\) have the same cost of a solution of R-\(\steinercycle\) after the addition of the new vertices. Given that equivalence, throughout this work, we assume that each terminal set in the \(\steinercycle\) contains exactly two vertices; i.e., we will refer to the R-SMCP as SMCP and denote terminal sets as terminal pairs.

\cite{LINTZMAYER2020134} proposed the first PTAS for the \(\steinercycle\). The algorithm is a randomized PTAS restricted to the Euclidean plane. In this context, the Euclidean \(\steinercycle\) encompasses a set of terminal pairs distributed across a plane and aim to calculate the line segments that connect the points with the least cost, considering that the same line segment might be crossed more than once in the cost function.

\cite{smcp_3apx} proposed a 3-approximation for complete metric graphs, which is based on a 2-approximation for the Survivable Network Design Problem and the concept of T-joins, a generalization of the well-known perfect matching. This algorithm is inspired by the results obtained for the metric Travelling Salesman Problem (TSP) by \cite{Christofides2022WorstCaseAO}.

It is also essential to notice that, even for the restricted case, the Steiner Multicycle Problem is \(\nonpoly\)-hard. This can be verified by reducing the TSP to the R-\(\steinercycle\) using a strategy similar to the one presented by \citeauthor{LINTZMAYER2020134}.

In this work, we propose a \textbf{polynomial-time approximation scheme} (PTAS) for the \(\steinercycle\) on \textit{bounded treewidth graphs}. From this result, we also present a PTAS on \textit{bounded genus graphs}. This class of graphs is a generalization of the well-known class of planar graphs.

% TODO deve ser ajustado após a troca de ordem dos capítulos.

This work is structured as follows. Chapter~\ref{chapter:definitions} introduces some basic definitions in Graph Theory and Approximation Algorithms, as well as some definitions used to present and prove the results of this work. Chapter~\ref{chapter:related_work} overviews the literature for \(\steinercycle\) and related problems. Chapters~\ref{chapter:pc-partition} and~\ref{chapter:spanner} introduce some tooling necessary for achieving the PTAS. Next, Section~\ref{section:ptas_bounded_tree} of Chapter~\ref{chapter:apx_schemes_for_smcp} proves a PTAS for bounded treewidth graphs, which is an essential result to prove the PTAS for bounded genus graphs in the Section~\ref{section:ptas_bounded_genus} of the same chapter. Chapter~\ref{chapter:experiments} presents some experimental results of the 3-approximation algorithm proposed by \cite{smcp_3apx}. Finally, Chapter~\ref{chapter:conclusion} delivers general remarks on this work and proposes possible paths for future work.

\chapter{Definitions}
\label{chapter:definitions}

This chapter presents the definitions and notations used throughout this work. The definitions shown are based on \cite{BondyNMurty} and \cite{Diestel}.

\section{Graph Notations and Concepts}

A graph is a pair \(G = (V, E)\) of sets satisfying \(E \subseteq [V]^2\); that is, the elements of \(E\) are subsets of \(V\) with two elements. The elements of \(V\) are called vertices and of \(E\) are called edges.

The set of vertices of a graph $G$ are also denoted as \(V(G)\) and the set of edges as \(E(G)\). We express membership of a vertex \(v\) to a graph \(G\) with the notation \(v \in V(G)\), and similarly, for an edge \(e\), we denote \(e \in E(G)\). Alternatively, we can use \(v \in G\) or \(e \in G\) when the context makes it clear that we are referring to vertices or edges, respectively.

A vertex \(v\) is \textbf{incident} to an edge \(e\) if \(v \in e\); so \(e\) is an edge connected to \(v\). The two vertices incidents to \(e\) are called the \textbf{endpoints} of \(e\). Another notation commonly used to represent an edge \(\{v, w\}\) is by \(vw\) simply, where \(v, w \in V(G)\) are the endpoints.

The number of vertices of a graph is referred to as the graph's \textbf{order}, denoted by~\(|G|\). The number of edges is referred to as its \textbf{size} and is denoted by \(||G||\).

Two vertices \(u\), \(v\) of \(G\) are \textbf{adjacent} or \textbf{neighbors} if there is \(e \in E(G)\) such that \(u\) and \(v\) are its endpoints. Equivalently, two edges \(e\) and \(f\) are adjacent if they have one of their endpoints in common.

Let \(v \in V(G)\) be a vertex. We denote as \(N(v)\) the \textbf{open neighborhood} of \(v\), that is, the set of all vertices \(w \in V(G)\) such that \(w\) is adjacent to \(v\). The \textbf{closed neighborhood} of \(v\), denoted as \(N[v]\) is \(N(v) \cup \{v\}\).

If there is an edge \(uv \in E(G)\) for any pair of vertices \(u, v \in V(G)\),  then we say that \(G\) is a \textbf{complete} graph. A complete graph of \(n\) vertices is denoted as \(K^n\).

The \textbf{degree} of a vertex \(v\) is the number of edges of \(G\) incident to \(v\), denoted by \(d(v)\). If \(v\) is a vertex such that \(d(v) = 0\), we say that \(v\) is an \textbf{isolated vertex}. The number \(\delta(G) := \min \{d(v) \colon v \in V(G)\}\) is the \textbf{minimum degree} of \(G\); the number \(\Delta(G) := \max \{d(v) \colon v \in V(G)\}\) is the \textbf{maximum degree} of \(G\).

We say that a graph \(H\) is a \textbf{subgraph} of \(G\) if \(V(H) \subseteq V(G)\) and \(E(H) \subseteq E(G)\), and denote this by \(H \subseteq G\). If for any pair of vertices \(v, w \in H \subset G\), the statement \(vw \in E(H)\) is valid if, and only if, \(vw \in E(G)\), then we say that \(H\) is an \textbf{induced subgraph} of \(G\), denoted as \(G[H]\). Moreover, we say that \(H\) is a \textbf{spanning subgraph} of \(G\) if \(V(H) = V(G)\).

A \textbf{path} in a graph \(G\) is a sequence of distinct vertices \(P = (v_1, v_2, \dots, v_k)\), \(P \subseteq V(G)\), such that, for every pair of consecutive vertices of \(P\), there is an edge in \(E(G)\) that connects these vertices. That is, for every \(v_i , v_{i+1}\) in \(P\), there is \(v_i v _{i+1} \in E(G)\). The number of vertices in the path is its \textbf{length}, and a path of length \(k\) is denoted by \(P^k\).

Let \(C = (v_1, v_2 \dots, v_k)\) be a sequence of adjacent vertices such that \(|C| \geq 3\) and the endpoints of the sequence \(C\) are equal, that is, \(v_1 = v_k\), then we say that \(C\) is a \textbf{cycle}. The \textbf{length} of a cycle equals the number of edges in it. 

A \textbf{walk} (of length \textit{k}) is a non-empty alternating sequence \((v_0, e_0, v_1, e_1, \dots, e_{k-1}, v_k)\) of vertices and edges such that \(e_i = \{v_i, v_{i+1}\}\) for all \(i < k\). In particular, in the case \(v_0 = v_k\), we call it a \textbf{closed walk}. Note that, in a walk, vertices and edges can be visited more than once.

% In this work, the term ``cycle'' is used to denote a closed walk in a graph or, equivalently, the multiset of edges that form the walk. A noteworthy observation is that all vertices along the walk have even degrees, taking into consideration the repetitions of edges. Conversely, we call ``simple'' a cycle without edges repetition.

Given \(u, v \in V(G)\), if there is a path between \(u\) and \(v\) in \(G\), we say that the vertices are \textbf{connected}.

If there is a partition of \(V(G)\) into subsets \(V_1 , V_2 , \dots, V_w\) such that two vertices \(u\) and \(v\) are connected if, and only if, \(u\) and \(v\) belong to the same subset \(V_i\), then we say that the subgraphs \(G[V_1], G[V_2], \dots, G[V_w]\) are the \textbf{components} of \(G\). If \(G\) contains a single component, we say that \(G\) is \textbf{connected}, otherwise \(G\) is \textbf{disconnected}.

Let \(e \in E(G)\). We define as \textbf{contracting} the operation on \(e\) which consists on replacing both endpoints of \(e\) with a new vertice \(v_e\) and connecting \(v_e\) to \(e\) endpoint's neighboors.

We denote the graph obtained from \(G\) by \textbf{contracting} the edge \(e \in E(G)\) at a new vertex \(v_e\), which becomes adjacent to all old neighbors of the endpoint vertices of \(e\), as \(G / e\).

Given a graph \(G\) and a subgraph \(H\) of \(G\), we \textbf{contract} \(H\) into a single vertex \(v\), generating a new graph \(G^\ast := G / H\). All vertices of \(G - H\) adjacent to at least two vertices from \(H\) are connected to \(v\) in \(G^\ast\) with multiple edges. We also define the process of \textbf{uncontracting} \(v\) as replacing it with the original subgraph \(H\) and reconstructing the previous connections with the neighboring vertices of \(H\) in \(G\).

Given a graph \(G\) and a graph \(H\), we say that \(H\) is a minor of \(G\) if, and only if, \(H\) can be obtained by a series of edges contractions on \(G\).

A collection \(\mathcal{S}\) is said to be \textbf{laminar} if and only if for any two sets \(C_1, C_2 \in \mathcal{S}\) we have \(C_1 \subseteq C_2\), or \(C_2 \subseteq C_1\), or \(C_1 \cap C_2 = \emptyset\). Suppose \(\mathcal{C}\) is a partition of a ground set \(V\). Then, \(\mathcal{C}(v)\) denotes, for each \(v \in V\), the set \(C \in \mathcal{C}\) that contains \(v\).

We define \(c \colon E(G) \to \mathbb{R}_\ge\), for \(e \in E(G)\) as the \textbf{cost} of \(e\). Naturally, \(c(H)\), for \(H \subseteq G\), is the sum of the cost of the edges of the subgraph \(H\). Let \(C = \{(v_1, v_2, \dots, v_k)\}\) be a cycle of \(G\). We define \(c(C) = \sum_{i=1}^{k-1} c(v_i v_{i+1})\).

For vertices \(v, w \in V(G)\) we define the \textbf{distance} between vertices $v$ and $w$, denoted by \(dist_G(v, w)\), as the cost of a shortest path between \(v\) and \(w\) that only uses edges of \(G\). If such a path does not exist, by convention, we consider \(dist_G(v, w) = +\infty\). We generalize the notation for distances between vertices and edges. Let \(v \in V(G)\) and \(uw \in E(G)\). Define the distance between the vertex \(v\) and the edge \(uw\) as \(dist_G(v, uw) := \min\{dist_G(v, u), dist_G(v, w)\} + c(uw)\).

A graph \(F\) without cycles is a \textbf{forest}. A connected forest is a \textbf{tree}. Let \(T\) be a tree and \(v \in T\) a vertex. If \(d(v) = 1\), we say that \(v\) is a \textbf{leaf} of \(T\), all other vertices are called inner vertices of \(T\). A tree is rooted if one of its vertices has been chosen as \textbf{root}. Naturally, a forest's component is a tree. 

Let \(T\) be a rooted tree with root \(r \in V(T)\), let \(v \in V(T)\). Note that there is only a single path between \(r\) and \(v\) in \(T\). Let \(P\) be this unique path between \(v\) and \(r\). We say that \(w \in V(T)\) is the \textbf{parent vertex} of \(v\) if \(w \in P\) and \(w \in N(v)\). Equivalently, we say that \(v\) is a \textbf{child vertex} of \(w\). Note that a vertex can have multiple children but only one parent.

Given a connected graph \(G\), we say that a \textbf{shortest-path tree} \(SPT\) rooted at a vertex \(v\) of \(V(G)\) is a spanning tree of \(G\) such that the path distance from root \(v\) to any other vertex \(u \in V(SPT)\) is a shortest path from \(v\) to \(u\) in \(G\), i.e., \(dist_{SPT}(u, v) = dist_{G}(u, v)\).

As defined by~\cite{ROBERTSON1986309}, a \textbf{tree decomposition} of \(G\) is a pair \((T, B)\) in which \(T\) is a tree and \(B = \{B_i \colon i \in V(T)\}\) is a family of subsets of \(V(G)\), so that:

\begin{itemize}
    \item \(\bigcup_{i \in V(T)} B_i = V(G)\);
    \item For each edge \(uv \in E(G)\), there is a \(i \in V(T)\) such that \(u, v \in B_i\);
    \item For each \(v \in V(G)\), the set of vertices \(\{i \in V(T) \colon v \in B_i\}\) forms a subtree of \(T\).
\end{itemize}

To differentiate from the original graph \(G\), we call the vertices of \(T\) \textbf{nodes}, where each node \(i\) has a corresponding \textbf{bag} of vertices \(B_i\). An example can be seen in Figure~\ref{fig:decomp1}.

\begin{figure}[H]
    \centering
    \includegraphics[scale=2]{imgs/decomp1.png}
    \caption{Example of a graph and its tree decomposition.~(\cite{imgTreeDecomp}).}
    \label{fig:decomp1}
\end{figure}

The \textbf{width} of a tree decomposition is the size of its largest bag minus one. The \textbf{treewidth} of a graph \(G\), denoted with \(tw(G)\), is the smallest width of a tree decomposition of \(G\).

To facilitate application in algorithms, we focus on a more restricted class of decompositions. We say that a tree decomposition \((T, B)\) is \textbf{nice} when \(T\) is a rooted tree and each \(i \in V(T)\) belongs to one of the following classes:

\begin{itemize}
    \item \(i\) is a \textit{leaf node} if it has no children;
    \item \(i\) is a \textit{union node} if \(i\) has exactly two children \(i_1, i_2\) and it holds \(B_i = B_1 = B_2\);
    \item \(i\) is an \textit{introduction node} if it has a single child \(i'\) and it holds \(B_i = B_{i'} \cup \{v\}\) for some vertex \(v \in V(G)\);
    \item \(i\) is a \textit{forgetting node} if it has a single child \(i'\) and it holds \(B_i = B_{i'} \backslash \{v\}\) for some vertex \(v \in V(G)\).
\end{itemize}

Given two bags \(B_i\) and \(B_j\), we say that \(B_j\) is a \textbf{descendant} of \(B_i\) if it is separated from the root bag by \(B_i\).

\section{Graph Planarity and Genus}

Informally, we characterize as \textbf{planar} a graph \(G\) that can be drawn on a plane without its edges intersecting outside their endpoints. Thus, the graph can be seen as points on a surface (vertices) with arcs (edges) connecting the points in such a way that arcs do not intersect outside a point. The surface region surrounded by arcs is called a \textbf{face} of the graph. Moreover, the outer region of the graph is the \textbf{outer face}, while all other faces are \textbf{inner faces} of \(G\). The \textbf{boundary} of \(G\) is the set of edges that separate the outer face of \(G\) from the rest of the graph, is denoted by \(\partial(G)\).

Formally,~\cite{Diestel} defines a planar graph as a pair \((V, E)\) of finite sets with the following properties:

\begin{itemize}
    \item \(V \subseteq \mathbb{R}^2\);
    \item every edge is an arc between two vertices;
    \item different edges have a different set of extreme points;
    \item The interior of an edge does not contain vertices or intersections with other edges.
\end{itemize}

For any planar graph, the set \(\mathbb{R}^2 \backslash G\) is open, and its regions are the faces of \(G\).

The intuitive idea of ``drawing'' a planar graph into a ``plane'' is what we call a \textbf{planar embedding} of \(G\). According to~\cite{Diestel}, an embedding in the plane of an (abstract) graph \(G\) is an isomorphism between \(G\) and a \textbf{plane graph} \(\tilde{G}\). The latter is referred to as \textbf{drawing} of \(G\).

\cite{BondyNMurty} showed that a planar embedding of a graph \(G\) exists if and only if \(G\) is also embeddable on the sphere. This idea will be useful to generalize the concept of embedding graphs to other surfaces.

Formally, they define a \textbf{surface} as a 2-dimensional manifold. We will skip a more detailed presentation of the topological definitions and properties of this structure, as for the reader it suffices to know that a sphere is a surface, as well as a torus and other constructions formed by pushing additional ``holes'' in the sphere. The number of ``holes'' in the sphere is what we call the \textit{genus} of the surface.

// TODO mencionar resultado que eh possivel determinar se um grafo eh planar em tempo polinomial

Informally, we say that the \textbf{genus} of a graph is the smallest number \textit{g} such that the graph can be drawn on a sphere with \textit{g} holes without edges crossing. From this idea, a planar graph has genus 0.

Interestingly, the problem of determining the genus of a graph is known to be \(\nonpoly\)-complete (\cite{THOMASSEN1989568}). However, as presented by \cite{LinearGenus}, the problem becomes treatable when the genus \textit{g} is fixed; that is, we can determine in polynomial time whether a graph can be embedded in a surface with genus \textit{g}, considering \textit{g} constant. As a secondary result, the authors also present a linear time algorithm, which computes the genus and constructs minimum genus embeddings of graphs of bounded treewidth. 

For a given planar graph \(G\), we say that \(H\) is its \textbf{dual} if there is a bijective function that maps the faces of \(G\) to the vertices of \(H\) and, for each edge \(e \in E(G)\) that separates the faces \(f\) and \(f'\) of \(G\), we have an edge that connects the vertices associated to \(f\) and \(f'\) in \(H\).

The concept of \textbf{faces} can be generalized to higher \textit{genus} graphs when we embed the graph on a surface of same genus.

\section{Classes of Optimization Problems}

In combinatorial optimization problems, we want to obtain the best solution from a finite, but potentially large, set of possible solutions. \cite{livroAprox} defined an optimization problem with the following elements: a set of instances, a set \(\sol(I)\) of viable solutions for a given instance \(I\), and a function \(\val (I, S)\) that associates to each instance and solution \(S \in \sol(I)\) a non-negative rational value. The solution to the optimization problem is the \(S\) that minimizes/maximizes \(\val(I, S)\).

More formally, for a given instance \(I\) there is a solution \(S^* \in \sol(I)\) such that \(\val(I, S^*) \leq \val(I, S)\) for all \(S \in \sol(I)\), considering a minimization scenario. The idea follows equivalently for a maximization problem. We denote \(\opt := \val(I, S^*)\) as the value of the optimal solution of \(I\). In particular, for the problems discussed in this work, we denote as \(\opt_{\mathcal{T}}(G)\) the value of an optimal solution in the graph \(G\) considering a set of pairs of terminals \(\mathcal{T}\).

There is an intimate relationship between the classes of optimization problems and the well-known classes of decision problems \(\poly\), \(\nonpoly\), \(\nonpoly\)-hard, \(\nonpoly\)-complete. There are optimization problems for which there are limitations of the approximability of their solutions by polynomial algorithms, considering \(\poly \neq \nonpoly\).

That said, optimization problems are classified according to their degree of approximability. By approximability, we observe a scenario in which we give up finding an optimal solution in favor of finding a solution that, while sub-optimal, can be found efficiently and maintain quality guarantees. We will consider and detail the classes PO, PTAS (and its variants QPTAS, FPTAS, and EPTAS), APX, and NPO, as described by \cite{livroAprox}.

According to \citeauthor{livroAprox}, the NPO class, which is an extension of the \(\nonpoly\) class for optimization problems, is composed of the problems where:

\begin{itemize}
    \item There is a polynomial function \textit{p} such that the size of the solution \(S\) is less or equal to the size of the instance \(I\) applied on \textit{p}, for every instance \(I\) of the problem and every feasible solution \(S\) of \(I\);
    \item There is a polynomial time algorithm that decides whether a given word is a valid representation of an instance of the problem;
    \item There is a polynomial time algorithm that decides whether a given object is a feasible solution for a given instance of the problem;
    \item There is a polynomial time algorithm that calculates \(\val(I, S)\), given \(I\) and \(S\).
\end{itemize}

We denote as PO the class of problems treatable in polynomial time. This class is an extension of the \(\poly\) class for optimization problems. Therefore, given a pro[blem \(\pi\), we say that \(\pi \in PO\) if there is a polynomial time algorithm that calculates an optimal solution for each instance \(I\) of \(\pi\). Examples of problems in this class are the Shortest Path and Minimum Spanning Tree problems.

Consider an optimization problem. Let \(A\) be an algorithm that for every instance \(I\) of the problem returns a feasible solution \(A(I)\) of \(I\). If the problem is one of minimization and \(val(I, A(I)) \leq \alpha \opt\) holds for every instance \(I\), then we say that \(A\) is a \textbf{\(\alpha\)-approximation} for the problem. We define \(\alpha\) as the \textbf{approximation ratio} of the algorithm.

The APX class comprises optimization problems in NPO for which there is an \(\alpha\)-approximation for some constant \(\alpha\). One of the most famous algorithms that provide this type of approximation is the Christofides Algorithm, developed by~\cite{Christofides2022WorstCaseAO} to the TSP problem restricted to metric graphs and which guarantees an approximation ratio of \(\alpha = 1.5\).

For a given optimization problem, we define as an \textbf{approximation scheme} an algorithm that receives an instance \(I\) of the problem as well as a parameter \(0 < \epsilon < 1\) and returns a solution \(S\) such that \(\val(I, S) \leq (1 + \epsilon) \opt\), for a minimization problem. We say that an algorithm is a \textbf{polynomial-time approximation scheme} (PTAS) if it has polynomial time for every fixed \(\epsilon\). Furthermore, an algorithm is a \textbf{fully polynomial-time approximation scheme}, or FPTAS, if its time is also polynomial in \(1 / \epsilon\). In contrast, a PTAS algorithm which is not FPTAS is exponential in \(1 / \epsilon\).

We will briefly comment on two other PTAS classes of interest. An \textbf{efficient polynomial-time approximation scheme} (EPTAS) is a more relaxed version of the FPTAS, where the complexity on \(1 / \epsilon\) does not need to be polynomial. I.e., a PTAS algorithm with time complexity \(\mathcal{O}(c^{1 / \epsilon} n^k)\), for \(c\) and \(k\) constants, is an EPTAS.

Another class worth mentioning is the \textbf{quasi-polynomial-time approximation scheme} (QPTAS), which allows a sub-linear factor as an exponent, i.e., an algorithm with time-complexity \(\mathcal{O}(c^{\log{n}})\) for every fixed \(\epsilon\) and where \(c\) is a constant.

From the definitions, it follows that \[PO \subseteq FPTAS \subseteq PTAS \subseteq APX \subseteq NPO.\]

\chapter{Related Work}
\label{chapter:related_work}


The Steiner Multicycle Problem (\(\steinercycle\)) is related to vehicle routing problems, particularly pickup and delivery problems. The literature for this class of problems is vast and considers multiple different restrictions and scenarios (\cite{surveyRouter}).

The \(\steinercycle\) stands between (and generalizes) two big classes of problems that are of much interest to the optimization and theoretical computing communities: the Traveling Salesman Problem (TSP) and the Steiner Tree Problem (STP).

We start this chapter by diving into the literature on the TSP, and some of its variants. We then move to the STP and discuss some of its variants that are more related to this study. 
Our literature review of these problems focuses on approximation algorithms, and more specifically on PTASes.

\section{Travelling Salesman Problem}

One of the most studied vehicle routing optimization problem is the Travelling Salesman Problem or TSP. In this problem, we aim to find a minimal-cost Hamiltonian cycle (i.e., a closed cycle that visits each vertex exactly once) in a given graph.

As mentioned in Chapter~\ref{chapter:introduction}, we can draw a link between the TSP and the \(\steinercycle\), since an instance of the TSP can be converted in polynomial time to an instance of the \(\steinercycle\). That is, the TSP is a particular case of the \(\steinercycle\).

The \textbf{Christofides Algorithm}, proposed by \cite{Christofides2022WorstCaseAO}, is one of the most famous approximation algorithms for the TSP. This algorithm is a \(3/2\)-approximation on metric graphs.

\cite{williamsonApxAlgs} present the following summary of the algorithm. Given a metric graph \(G\), we calculate its minimum spanning tree \(M\). Let \(O\) be the set of odd-degree vertices in \(M\). By the Handshaking Lemma, there are an even number of odd-degree vertices in \(M\). One of the classic results in combinatorial optimization is that given a complete graph (on an even number of vertices), it is possible to calculate a perfect matching of minimum total cost in polynomial time. Considering that, we can calculate it for \(O\). If we add the edges of the perfect matching over \(O\) in \(M\), we create an Eulerian multigraph \(H\) since it is connected and all vertices have even degrees. Finally, we can shortcut the duplicated edges to create a solution of no greater cost corresponding to a Hamiltonian cycle.

Interestingly, more than four decades of research failed to improve upon \citeauthor{Christofides2022WorstCaseAO}' \(3/2\) factor, until \cite{slightlyBetterApxTSP} provided a randomized \((3/2 - \epsilon)\)-approximation for \(\epsilon > 10^{-36}\). Their method follows similar principles to \citeauthor{Christofides2022WorstCaseAO}' algorithm but uses a randomly chosen tree from a carefully selected random distribution in place of the minimum spanning tree.

It is generally known that finding PTASes is difficult for a lot of problems. In particular, \cite{noPTASMetricTSP} showed that there is no PTAS for metric TSP, as well as for the Steiner Tree Problem, Maximum Directed Cut Problem (in which we want to find a bipartition of the vertices of the graph such that the cost of the edges crossing the partition is maximum), and Shortest Superstring Problem (in which we want to find a shortest possible string that contains every string in a given set as substrings), unless \(\poly = \nonpoly\).

\cite{PTASeuclidianTSP} designed a PTAS for \textbf{Euclidean TSP}. They strategy consisted in recursively partitioning the plane (using a randomized variant of the quadtree) such that some \((1 + \epsilon)\)-approximate salesman tour crosses each line of the partition at most \(\mathcal{O}(1/\epsilon)\) times. Such a tour is found by dynamic programming. For each line in the partition, the algorithm first ``guesses'' where the tour crosses this line and the order in which those crossings occur. Then it recurses independently on the two sides of the line. Only \(n \log{n}\) distinct regions are in the partition, where \(n\) is the number of nodes in the plane. Furthermore, the “guess” can be fairly coarse, so the algorithm spends only \(\mathcal{O}(\log{n})^{\mathcal{O}(1/\epsilon)}\) time per region, for a total running time of \(n \cdot (\log{n})^{\mathcal{O}(1/\epsilon)}\). The result still holds even if the nodes are in \(\mathbb{R}^d\), but the running time increases to \(\mathcal{O}(n (\log{n})^{(\mathcal{O}(\sqrt{dc}))^{d-1}})\).

One of the few practical implementations of a PTAS in the literature was done by \cite{implementationPTASeuclidianTSP} for \citeauthor{PTASeuclidianTSP}'s algorithm. Their work describes an implementation of the Euclidean TSP that is based on the essential steps of \citeauthor{PTASeuclidianTSP}'s PTAS with some additional heuristics to improve the running time. In general, their results showed that this PTAS can achieve good performance in practice, although due to challenges during implementation, the authors had to make decisions that caused the loss of theoretical guarantees of the solution's quality. This illustrates how challenging implementing a PTAS can be.

\cite{baker1994} presented a method for obtaining PTASes for a variety of optimization problems in planar graphs, e.g. maximum-weight independent set and minimum-weight vertex cover. This method was generalized by \cite{demaine2005} to derive algorithms for nonlocal problems, such as feedback vertex set and connected dominating set. They were able to derive approximation schemes for subclasses of minor-excluded graphs that involve turning the input graph into a low treewidth graph. That way, their results apply to graphs that are not planar.

\cite{KleinTSP} improved the results from the work of \cite{basicPTASplanarTSP} by generating an EPTAS for TSP in planar graphs. 
To do so, the authors proposed a framework based on the results from \cite{baker1994} and \cite{demaine2005}, which consists of the following steps.

The first step creates a subgraph \(H\) called \textbf{spanner}.
A spanner \(H\) has two properties of interest: the Quasi-optimality property, which means there is a solution of cost \((1 + \epsilon) \opt\) in \(H\) and the Shortness property, which means the cost of \(H\) is bounded by a constant times \(\opt\). The second step is to find a set of edges of cost at most \(\mathcal{O}(\epsilon \opt)\) contained in \(H\), then proceed to contract those edges. According to~\cite{Demaine2010}, the resulting graph of this process has bounded treewidth. The third step uses a dynamic programming approach on this bounded treewidth graph to obtain an optimal solution. Finally, the solution is the union of the resulting graph from the dynamic programming with the set of edges that was contracted.

\cite{KleinTSP} also showed that this framework could be generalized to obtain PTASes for multiple problems in planar graphs. \cite{Bateni} even commented: ``Most approximation schemes for planar graph problems use (implicitly or explicitly) the fact that the problem is easy to solve on bounded treewidth graphs''. That fact is the basis for his work on the Steiner Forest Problem (which will be presented later in this chapter).

\cite{EPTASeuclidianTSP} improved on \cite{PTASeuclidianTSP} work by using ``light'' spanners for Euclidean graphs to obtain an EPTAS for Euclidean TSP.~\cite{contraction-decomposition-in-h-minor-free-graphs} proved that any graph excluding a fixed minor can have its edges partitioned into a desired number \(k\) of color classes such that contracting the edges in any one color class results in a graph of treewidth linear in \(k\). This result allowed them to generalize \citeauthor{KleinTSP}'s framework to H-minor-free graphs. Given that, and using the spanners proposed by~\cite{light_spanners_tsp}, \citeauthor{contraction-decomposition-in-h-minor-free-graphs} proposed the first PTAS for TSP in weighted H-minor-free graphs. This result improved upon the early work by~\cite{light_spanners_tsp}, who designed a QPTAS for the same problem.

Since \citeauthor{light_spanners_tsp}'s spanner has weight \(\mathcal{O}(\log{(n)} \; \mathrm{poly}(1 / \epsilon)) \opt\), \citeauthor{contraction-decomposition-in-h-minor-free-graphs}'s PTAS is not efficient.~\cite{eptas-tsp-h-minor-free} improved on the spanner's size resulting in an EPTAS for the problem. The authors mentioned that designing a PTAS for the Subset TSP (also known as Steiner Cycle Problem) in H-minor-free graphs is a central open problem of the field (we will present more on this ahead).

Two related problems to the TSP are the \textbf{Steiner Cycle Problem} and the \textbf{Multiple Steiner Cycle Problem}. These will be presented in the next section.

\section{Steiner-related problems}

We loosely define a ``Steiner-related'' problem as any problem that aims at making some kind of connection between terminals inside one or more sets of vertices. Traditionally, these connections are made with trees or cycles.

During this section, we will present some relevant results in the literature for a few Steiner-related problems, in particular the Steiner Tree Problem, Steiner Cycle Problem, the Steiner Forest Problem, and the Multiple Steiner Traveling Salesman Problem. At the end of the section, we will present some results for the \(\steinercycle\).

\subsection{Steiner Tree Problem}

The Steiner Tree Problem (STP) is one of the most studied problems in the combinatorial optimization literature. The problem consists of finding a tree with minimum cost that connects a subset of the vertices, called terminals, in the graph. It was one of the \(\nonpoly\)-complete problems presented by~\cite{Karp1972}.

In their work,~\cite{Borradaile2009b} drew inspiration from~\cite{KleinTSP} framework to create a PTAS for the Steiner Tree Problem in planar graphs. The authors built a spanner using a subgraph \(MG\), called Mortar Graph, and a set of \textbf{bricks} - subgraphs that are only connected to the rest of the graph via a bounded set of vertices called \textbf{portals}. More details of those concepts are given ahead. This spanner would suffice to be applied to \citeauthor{KleinTSP}'s framework to obtain a PTAS, but it was noted by the authors that such PTAS would yield a time complexity double exponential on \(1 / \epsilon\).

To improve this result, the authors decomposed \(MG\) into a set of subgraphs called \textbf{parcels}. This decomposition was done via a breadth-first search in the dual of \(MG\). As a result, each parcel is an embedded planar subgraph of \(MG\). From this, the authors defined a simple graph called \textbf{parcel graph} in which each vertex represents a parcel in \(MG\).

The constructed parcel graph has some interesting properties, one of which is that the planar dual of each parcel has a spanning tree with depth limited by a constant. That way, they can compute the optimal Steiner Tree for each parcel and take their union to obtain a solution for the original graph. This new approach resulted in a running time singly exponential in \(\mathrm{poly}(1 / \epsilon)\).

To use the Mortar Graph to build the spanner, the authors presented and demonstrated a structural theorem that guarantees that the optimal solution found in the Mortar Graph, and its bricks, is at most \((1 + \epsilon) \opt\).

\subsection{Steiner Cycle Problem}

The \(\steinercycle\) was also studied in a more restricted variant, where we must cover all terminals with a single cycle. This more restricted problem is known in the literature as \textbf{Steiner Cycle Problem}, SCP (it is also known as \textbf{Steiner TSP} or \textbf{Subset TSP}). It is possible to observe that the SCP is equivalent to the TSP in the scenario where all vertices are terminals.

\cite{Cornuejols1985} first introduced the SCP (named Steiner TSP by the authors) while studying the problem in its graphical case and investigating its polyhedron in series–parallel graphs.

\cite{SalazarSteinerCycle} analyzed the polyhedral structure associated with the Steiner Cycle Problem and introduced two lifting strategies to generate inequalities facet defining based on the TSP polytope.

\cite{SteinovaSteinerCycle} presented a 3/2-approximation for the SCP in metric graphs. The authors also showed that there is no constant time approximation for general graphs unless \(\poly = \nonpoly\). This result implies that the SCP, and by consequence the \(\steinercycle\), is NPO-hard in general graphs.

\cite{Arora1998APA} found a \textit{quasipolynomial-time approximation scheme} for the SCP in planar graphs. To find a \((1 + \epsilon)\)-approximated solution for the problem, the proposed algorithm requires time \(\mathcal{O}(n^{\mathrm{poly}(\log n, 1/\epsilon)})\). They made the following conjecture that allows the derivation of a PTAS from this result: ``There exists a function \(f(\cdot)\) such that given \(\epsilon > 0\), a planar graph \(G\) with edge-weights, and a subset \(S\) of vertices, there exists an edge-induced subgraph \(G'\) that \((1 + \epsilon)\)-approximates all distances between nodes in \(S\), and furthermore, \(G'\) has total edge weight at most \(f(\epsilon)\) times the minimum Steiner tree weight for \(S\).''

\cite{klein2006} created a PTAS for SCP in planar graphs, the first one proposed for the problem, by following up from \cite{Arora1998APA}. They confirmed \citeauthor{Arora1998APA}'s conjecture to be true by showing a constructive proof that implies a polynomial-time algorithm for the construction of the subgraph \(G'\), i.e., the \textbf{spanner}.

\cite{klein2014} proposed a sub-exponential parameterized algorithm for the problem with time complexity \((2^{\mathcal{O}(\sqrt{k} \log{k})} + W) \cdot n^{\mathcal{O}(1)}\), where \(n\) is the number of vertices and \(k\) is the numbers of terminals, if \(G\) is a planar graph with weights that are integers no greater than \(W\). The primary strategy to achieve this result consists of two steps: (1) find a locally optimal solution and (2) use it to guide a dynamic program. The proof of correctness of the algorithm depends on the treewidth of a graph obtained by combining an optimal solution with a locally optimal solution.

\cite{LeSubsetTspPTAS_H_MinorFree} proposed a PTAS for the SCP in H-minor-free graphs, for any fixed graph \(H\), expanding on \cite{eptas-tsp-h-minor-free} work. Their main contribution is the concept of a nearly light subset spanner construction based on sparse spanner oracles. They show that spanner oracles with weak sparsity are necessary and sufficient to construct light subset spanners, even for general graphs.

This result is interesting for several reasons. Despite the many advances in meta-algorithms related to PTAS in H-minor-free graphs, a PTAS for SCP on this class of graphs was still unknown. Moreover, there is evidence that a PTAS for this problem is impossible in more general graphs than H-minor-free.


\subsection{Multiple Steiner Traveling Salesman Problem}

The \textbf{Multiple Steiner Traveling Salesman Problem with Order Constraints}, or briefly MSTSPO, is a close variant of the Steiner Multicycle Problem. The difference is that the MSTSPO fixes a set of \(K\) salesman, each having to visit a set of terminals to form the cycles, and considers that each cycle must visit the terminals in a predefined order.

% S. Borne, A. R. Mahjoub, and R. Taktak. A branch-and-cut algorithm for the multiple steiner  TSP with order constraints.
\cite{BORNE2013487} formulated the problem motivated by survivability issues in multilayer telecommunication networks as Multiple Steiner TSP with Order Constraints. Their work proposes an integer linear programming formulation for the problem and investigates the associated polytope. They also present new valid inequalities and discuss their facial aspect. Finally, they devised a Branch-and-cut algorithm and presented preliminary computational results.

% The Multiple Steiner TSP with order constraints: complexity and optimization algorithms
\cite{Gabrel2020} expanded on \citeauthor{BORNE2013487}'s work by proposing a few integer programming formulations and both Branch-and-Cut and Branch-and-Price algorithms to solve the problem. They presented extensive computational results, showing the efficiency of the algorithms.

\subsection{Steiner Forest Problem}

Another \(\steinercycle\) related problem is the Steiner Forest Problem (SFP), where instead of using cycles to connect terminal pairs, we restrict ourselves to using only trees. In a similar way that the \(\steinercycle\) generalizes the SCP, the SFP generalizes the STP. 

\cite{Bateni} proposed a PTAS for the Steiner Forest Problem based on the work of \cite{Borradaile2009b}. To do so, the authors presented a clustering algorithm called \textbf{Prize Collecting Clustering}, which aims at segmenting the original graph in a set of trees \(\{T_1, \dots, T_k\}\) in such a way that the sum of the cost of all trees is at most \((4/\epsilon + 2) \opt\), where \(\opt\) is the cost of an optimal solution for the entire graph, encompassing all terminal sets.

Besides that, considering \(\mathcal{D}_i\) as a set of terminals that must be connected, and \(T_i\) as the tree that connects the terminals in \(\mathcal{D}_i\), the sum of the costs of all the minimal Steiner Forests, connecting only the vertices within each \(\mathcal{D}_i\), totals at most \((1 + \epsilon) \opt\). That can be expressed as \(\sum_i \opt_{\mathcal{D}_i} \leq (1 + \epsilon) \opt\).

In simpler terms, the union of optimal solutions for each terminal set forms an \(1 + \epsilon\)-approximated solution for the input graph. Finally, for each tree \(T_i\) obtained by the clustering, the authors applied the framework proposed by \cite{KleinTSP}, and expanded by \cite{Borradaile2009b, Borradaile2012}.

\subsection{Steiner Multicycle Problem} \label{subsection:steinermulticycle}

\cite{LINTZMAYER2020134} proposed the first PTAS for the \(\steinercycle\). The algorithm is a randomized PTAS restricted to the Euclidean plane. In this context, the Euclidean \(\steinercycle\) encompasses a set of terminal pairs distributed across a plane and aim to calculate the line segments that connect the points with the least cost, considering that the same line segment might be crossed more than once.

The algorithm groups the terminal pairs in such a way that pairs that are far away from each other belong to different groups. The authors guarantee that the union of the optimal solution calculated in each group is bounded by a constant times the optimal solution when considering all terminal pairs. Then, for each group of terminal pairs, they create a square called \textbf{root dissection square}, which is a bounding box containing all terminal pairs of the group. This square has a cost of at most a constant times the optimal solution cost considering the terminals pairs in the square.

For each square, they run a process of recursive dissection. In this process, each square is subdivided into four squares of equal size, using vertical and horizontal lines. They select a limited number of points in its border as portals, for each generated square during the partitioning. That way, the solutions generated by the algorithm cross the squares only through the portals. At the end of the process, each square that has not been partitioned is broken into cells, like in a grid. 

For each square \(R\), the authors define a memoization table \(M_R\), indexed by valid configurations of \(R\), in other words, valid subpartitions of the cells and portals of \(R\). Considering a valid configuration of \(R\) called \(\pi\), \(M_R(\pi)\) is the cost of the minimal solution that is \textbf{compatible} with \(\pi\) and \textbf{conforms} to \(R\). A solution is compatible with \(\pi\) when its line segments respect the partition of \(\pi\) and a solution conforms with \(R\) when it is feasible, connects the necessary terminal pairs, only crosses the borders in \(R\) through portals, and makes that crossing a limited number of times.

To write an entry in \(M_R(\pi)\), the authors observe all possible configurations of the squares generated by subdividing \(R\), the ``children'' squares of \(R\), and only consider entries consistent with \(\pi\) from all these configurations. They show that the combined solutions of the four ``children'' squares generate a solution compatible with \(\pi\) and conform to \(R\).

\cite{smcp_3apx} presented a 3-approximation for \(\steinercycle\) on metric and complete graphs. The algorithm was based on the work done by \cite{Christofides2022WorstCaseAO} for the metric TSP and consists of the following three basic steps.
The first step is to perform a 2-approximation for the Survivable Network Design Problem (SNDP), considering a specific set of parameters which will be detailed next. The authors described the SNDP as follows: given a graph \(G\), a weight function \(w: E(G) \rightarrow \mathbb{Q}_+\), and a non-negative integer \(r_{ij}\) for each pair of vertices \(i\), \(j\) with \(i \neq j\), representing a connectivity requirement, the goal is to find a minimum weight subgraph \(G'\) of \(G\) such that, for every pair of vertices \(i, j \in V(G)\) with \(i \neq j\), there are at least \(r_{ij}\) edge-disjoint paths between \(i\) and \(j\) in \(G'\).

They also observed that given an \(\steinercycle\) instance, it is easy to define an equivalent SNDP instance by considering \(r_{ij} = 2\) for any vertices \(i\) and \(j\) that belong to the same terminal set and \(r_{ij} = 0\) otherwise. That way, the optimum value of the SNDP is also a lower bound on the optimum for the Steiner Multicycle Problem: indeed, an optimal solution for the Steiner Multicycle Problem is a feasible solution for the SNDP with the same cost. Those parameters of the SNDP are used on the first step of the algorithm.

Let \(T\) be a set of vertices that has an even number of elements in \(G\). A set \(J\) of edges in \(G\) is a \textbf{T-join} if the set of vertices of \(G\) that are incident to an odd number of edges in \(J\) is exactly \(T\).

Let \(G'\) be the output of the 2-approximation for the SNDP. Consider \(T\) to be the set of odd-degree vertices in \(G'\). Thus, the second step of the algorithm consists of calculating in polynomial time a minimum \(T\)-join \(J\) in \(G'\). Finally, as the third step, we obtain an Eulerian graph \(H\) from \(G'\) by doubling the edges in \(J\) and, by shortcutting an Eulerian tour for each component of \(H\), one obtains a collection \(C\) of cycles in \(G\) that is the output of the algorithm.


\cite{Borradaile2012} expanded the works of \cite{Borradaile2009b} and \cite{KleinTSP} to obtain PTASes in subset connectivity problems in bounded genus graphs. To do that, the authors proposed two methods. The first method is a generalization of the Mortar Graph framework by \cite{KleinTSP}. The second applies the strategy proposed in \cite{Borradaile2009b}, but using the simpler tree decomposition instead of the parcel decomposition.

In this work, we intend to expand on the results found by \cite{LINTZMAYER2020134} by proposing a PTAS for the \(\steinercycle\) in graphs with bounded genus. To that end, we will use various techniques proposed by \cite{Bateni} and \cite{Borradaile2009b} for the Steiner Forest and Steiner Tree problems, respectively, such as the Mortar Graph, the Prize Collecting Clustering, and the spanner.

\chapter{Computational Experiments}
\label{chapter:experiments}

We ran computational experiments in two distinct algorithms on instances for the Steiner Multicycle Problem (SMCP). In particular, we implemented the \(3\)-approximation algorithm proposed by \cite{smcp_3apx}, and a relaxed solver to obtain a lower bound to optimal results for the same instances.

It is worth mentioning that we initially tried to run an exact solver to obtain the optimal solutions, but the execution time became prohibitive for this strategy.

# TODO add as footnote
All experiments were executed in an Intel Core i5 11400H with a 16Gb 3200MHz RAM. The code was compiled with Clang version 15.0.1 on a Windows 10 machine. You can see the code at: https://github.com/RaulWCosta/steiner-multicycle-problem-3-apx

The algorithms were implemented in C++ using the Graph library \cite{lemon} and the \cite{gurobi}.

The linear relaxation was implemented using \cite{Pereira2018TheSM} formulation. From \citeauthor{Pereira2018TheSM}, given an instance \((G, c, T)\) of the SMCP, consider a function \(f: 2^V \rightarrow \{0, 1\}\) such that for each non-empty set \(S \subset V\) we have \(f(S) = 1\) if, and only if, there is some \(T_a \in \mathcal{T}\) such that \(S \cap T_a \neq \emptyset\) and \(T_a \nsubseteq S\), i.e., \(S\) is a cut that separates terminals in \(T_a\). They use \(i \in T\) to denote a vertex in a terminal. The SMCP can be formulated as:

\begin{align}
&\text{minimize} &\sum_{e\in E} c_e x_e \label{form-pereira:1}\\
&\text{subject to} &\sum_{e \in \delta(i)} x_e &= 2 &&\text{\(\forall i \in \mathcal{T}\)} \label{form-pereira:2}\\ 
&&\sum_{e \in \delta(S)} x_e &\geq 2 f(S) &&\text{\(\emptyset \neq S \subset V\)} \label{form-pereira:3}\\ 
&&x_e &\in \{0, 1, 2\} &&\text{$e \in E(G)$} \label{form-pereira:4}
\end{align}

Where \(\delta(S)\) denotes the set of edges having exactly one endpoint in \(S\). The variable \(x_e\) indicates if an edge is used in the solution, Constraint~\eqref{form-pereira:2} assures that exactly one cycle covers each terminal, Constraint~\eqref{form-pereira:3} assures that vertices belonging to a terminal set \(T_a \in \mathcal{T}\) are connected, and Constraint~\eqref{form-pereira:4} allows each edge to be used at most twice.

By relaxing the integrality Constraint~\eqref{form-pereira:4} we obtain the linear program used in the implementation. Note that the number of constraints~\eqref{form-pereira:3} is exponential. However, we managed to compute the relaxed LP in polynomial time by using Gomory-Hu trees to solve maximum flow problems between each pair of vertices in the same terminal set \(T_a\). This is the same strategy employed by \cite{Pereira2018TheSM} on their implementations.

# TODO parei aqui

\section{3-approximation Algorithm}

\cite{smcp_3apx} proposed the following algorithm to obtain a 3-approximation for SMCP.

\begin{algorithm}
\caption{SMCP 3-approximation}
\label{algorithm:smcp-3-apx}
\begin{algorithmic}[1]

\Require A complete graph \(G\), a weight function \(w: E(G) \rightarrow \mathbb{Q}_+\) satisfying the triangle inequality, and a partition \(\mathcal{T} = \{T_1, \dots, T_k\}\) of \(V(G)\)

\Ensure A collection \(\mathcal{C}\) of cycles that respects \(\mathcal{T}\)

\State \(r_{i, j} \gets 2\) for every \(i, j \in T_a\) for some \(1 \leq a \leq  k\)
\State \(r_{i, j} \gets 0\) for every \(i \in T_a\) and \(j \in T_b\) for some \(1 \leq a < b \leq k\).
\State \(G' \gets 2ApproxSND(G, w, r)\) \label{alg:3-apx:snd-2-apx}
\State Let \(T\) be the set of odd degree vertices in \(G'\)
\State Let \(w'\) be the restriction of \(w\) to the edges in \(G'\)
\State \(J \gets MinimumTJoin(G', w', T)\) \label{alg:3-apx:t-join}
\State \(H \gets G' + J\)
\State \(\mathcal{C} \gets ShortCutting(H)\) \label{alg:3-apx:shortcutting}
\State return \(\mathcal{C}\)


\end{algorithmic}
\end{algorithm}


% \begin{enumerate}
%     \item Perform a 2-approximation for the Survivable Network Design Problem (SNDP) in \(G\) to find a subgraph \(G'\)
%     \item Calculate a minimum \(T\)-join \(J\) in \(G'\).
%     \item Create an Eulerian graph \(H\) by doubling each edge of \(J\) and adding them to \(G'\).
%     \item Shortcut an Eulerian tour for each component of \(H\) to obtain a collection \(C\) of cycles in \(G\), which is the algorithm's output.

% \end{enumerate}

To calculate the SNDP in line~\eqref{alg:3-apx:snd-2-apx}, we implemented the 2-approximation proposed by \cite{snd-2-apx}. This algorithm solves the linear relaxation of the problem and rounds the solutions. We also used Gomory-Hu trees to add cuts to the linear relaxation at each iteration.

As mentioned by the authors, the Theorem~1 (from \cite{smcp_3apx}) implies that a minimum weight perfect matching in the graph \(G[T]\) weights at most \(w(G')/2\). This means that one can exchange line~\eqref{alg:3-apx:t-join} to compute, instead, a minimum weight perfect matching \(J\) in \(G[T]\). We applied this strategy in our implementation.

The shortcutting in line~\eqref{alg:3-apx:shortcutting} is done by finding any Eulerian cycle within each component of \(H\). We conjecture that the quality of the 3-approximation could be further improved by adopting a better strategy for shortcutting, at the cost of a potentially greater execution time.

\section{Instances}

As of the time of writing, the only experimental study available for the SMCP (more specifically, its restricted version R-SMCP) was the one presented by \cite{Pereira2018TheSM}. In their work, they evaluated implementations of their proposed algorithm and a well-known metaheuristic using two instance types.

Type 1 is a set of instances from the multi-commodity one-to-one pickup-and-delivery traveling salesman problem (m-PDTSP), and type 2 is a set of randomly generated instances created by them.

\cite{HERNANDEZPEREZ2009987} generated a set of instances to the m-PDTSP. For each instance, they generated \(2n - 2\) uniformly random points with coordinates from \(-500\) to \(500\), a vertex in position \(0\) with coordinates \((0, 0)\) and a vertex in position \(2n - 1\) also with coordinates \((0, 0)\) (corresponding to Class 3 of \cite{HERNANDEZPEREZ2009987}). Like \cite{Pereira2018TheSM}, we only consider the vertex distribution of these instances. For each \(i \in \{0, \dots, n - 1\}\), we assign a vertex \(i\) as a pickup point of an agent and \(i + n\) as its corresponding delivery point. The instances have 6, 11, and 16 agents, totaling 210 instances. This set of instances is of the type 1.

\cite{Pereira2018TheSM} generated a set of instances having 16, 32, 64, 128, and 256 agents, where vertices correspond to points distributed in a square of dimensions \(100000\) by \(100000\). The square is divided into \(1 \times 1\), \(2 \times 2\), \(3 \times 3\), \(4 \times 4\), and \(5 \times 5\) frames, and each pair of pickup and delivery is in the same frame. The space between frames, which we call a wall, has \(0\%\), \(10\%\), \(20\%\), \(30\%\), or \(40\%\) of the frame’s size. The wall can be seen as a rectangle separating the different frames (see Figure~\ref{fig:instances_type_2}). Notice that there is no wall for the instances with division \(1 \times 1\). The location of each point is chosen uniformly at random. For each combination of the number of agents, frames, and wall size, three instances were generated with different seeds, therefore, 315 instances of type 2 were generated.

\begin{figure}[H]
    \centering
    \includegraphics[scale=0.5]{imgs/instances_type_2.png}
    \caption{Instances randomly generated with 16 agents, in (a) a 1 x 1 frame with no wall, in (b) a 3 x 3 frame with \(0\%\) wall, and in (c) a 3 x 3 frame with \(30\%\) wall.}
    \label{fig:instances_type_2}
\end{figure}

\section{Computational Results}

A summary of the experiment results broken by instance group can be seen in Table~\ref{table_avg_apx}. The table shows the gap between the result achieved by the 3-approximation and the linear relaxation, as well as, its respective execution times. The first line of the table shows the average results from the type 1 instances. The remaining table sub-divides the type 2 instances depending on some properties. The ``\(1 \times 1\)'' class contains results with a single frame, while the ``\(2 \times 2\)'' class contains the results of the instances with 4 frames, etc. The ``\(W0.x\)'' class contains the results of the instances with wall separation of \((10 \times x)\%\) of the size of the frame (as illustrated in the Figure~\ref{fig:instances_type_2} (c)). The last lines of the table separate the instances by the number of ``agents'' (which is how \citeauthor{Pereira2018TheSM} call terminal pairs).

\begin{table}[H]
\caption{Summary of results by class}
\centering
    \begin{tabular}{lrrrr}
        \toprule
        Classes & num Inst & GAP (\%) & APX time (s) & relaxed solver time (s) \\
        \midrule
        m-PDTSP & 210 & 33.80 & 0.00 & 0.00 \\
        \hline
        1 $\times$ 1 & 15 & 41.06 & 159.34 & 0.09 \\
        2 $\times$ 2 & 75 & 40.67 & 46.45 & 0.09 \\
        3 $\times$ 3 & 75 & 36.46 & 45.01 & 0.10 \\
        4 $\times$ 4 & 75 & 32.89 & 64.33 & 0.09 \\
        5 $\times$ 5 & 75 & 30.77 & 100.85 & 0.09 \\
        \hline
        W0.0 & 75 & 36.90 & 145.25 & 0.09 \\
        W0.1 & 60 & 37.78 & 109.89 & 0.09 \\
        W0.2 & 60 & 35.36 & 55.96 & 0.09 \\
        W0.3 & 60 & 33.09 & 7.33 & 0.09 \\
        W0.4 & 60 & 32.72 & 5.91 & 0.09 \\
        \hline
        rg-016 & 63 & 23.76 & 0.00 & 0.00 \\
        rg-032 & 63 & 26.02 & 0.03 & 0.00 \\
        rg-064 & 63 & 34.63 & 0.56 & 0.02 \\
        rg-128 & 63 & 36.74 & 12.44 & 0.08 \\
        rg-256 & 63 & 40.98 & 330.44 & 0.35 \\
        \hline
        total average & & 34.21 & 197.10  & 0.29 \\
        \bottomrule
    \end{tabular}
\label{table_avg_apx}
\end{table}


We can see a positive trend between the number of agents (i.e., terminal pairs) and the solution gap. It is also possible to observe a negative trend between the number of frames and the quality of the solution, indicating that the algorithm does not handle well terminal pairs whose terminals are far from each other.

The relationship between the number of terminals in each instance and the optimality gap of the algorithm can be seen in Figure~\ref{fig:hist_opt_gap}. The figure shows that for all instances, the algorithm got significantly closer to the optimality than its 3-approximation guarantee. Although the results, accounting for both execution time and solution quality, were inferior to the heuristic proposed by \cite{Pereira2018TheSM}. 

% \begin{figure}[t!]
%     \centering
%     \includegraphics[scale=0.4]{imgs/hist_opt_gap2.png}
%     \caption{Number of instances per optimality gap. An optimality gap of \(1.6\) means that the solution cost is \(60\%\) more than the cost of the relaxation.}
%     \label{fig:hist_opt_gap}
% \end{figure}

\begin{figure}[t!]
\centering
\begin{tikzpicture}
\begin{axis}[
    ymin=0, ymax=80,
    minor y tick num = 3,
    area style,
    ]
\addplot+[ybar interval,mark=no] plot coordinates { (1.0476854749768996, 11) (1.0853661041789102, 14) (1.123046733380921, 33) (1.1607273625829317, 30) (1.1984079917849424, 52) (1.2360886209869533, 55) (1.273769250188964, 36) (1.3114498793909748, 63) (1.3491305085929854, 60) (1.386811137794996, 43) (1.424491766997007, 37) (1.4621723961990176, 36) (1.4998530254010283, 11) (1.5375336546030391, 7) (1.5752142838050498, 21) (1.6128949130070604, 1) (1.6505755422090713, 7) (1.688256171411082, 1) (1.7259368006130926, 0) (1.7636174298151035, 7) };
\end{axis}
\end{tikzpicture}
\caption{Number of instances per optimality gap. An optimality gap of \(1.6\) means that the solution cost is \(60\%\) more than the cost of the relaxation.}
\label{fig:hist_opt_gap}
\end{figure}

Although the algorithm tended to perform better on smaller instances, from the 10 worst performance instances (that is, with the largest gap from the linear relaxation), 9 are from the instances of type 1 containing only 6 terminal pairs.

The implementation of the 3-approximation algorithm executes the shortcut step in line~\eqref{alg:3-apx:shortcutting} by naively finding any Euclidean subgraph contained in each component of the solution; we conjecture that the quality of this algorithm could be improved by applying a more robust shortcutting strategy on the cost of a greater execution time.

In all instances, the time spent to calculate the 2-approximation for the Survivable Network Design Problem it accounted for more than \(99\%\) of the total execution time of the algorithm. Moreover, the iterative calculation of Gomory-Hu trees, a step during the solving of the SNDP, has a significant impact on the total execution time, especially for bigger instances, as we can see in Table~\ref{table_gomory_hu_gap}.

\begin{table}[t!]
\caption{Participation of Gomory-Hu trees calculation on total execution time}
\label{table_gomory_hu_gap}
\centering
    \begin{tabular}{lrrrr}
        \toprule
        num vert & num inst & min (\%) & mean (\%) & max (\%) \\
        \midrule
        12 & 70 & 0.00 & 0.00 & 0.00 \\
        22 & 70 & 0.00 & 0.00 & 0.00 \\
        32 & 133 & 0.00 & 0.00 & 0.00 \\
        64 & 63 & 0.00 & 13.63 & 40.90 \\
        128 & 63 & 23.31 & 39.64 & 75.80 \\
        256 & 63 & 30.30 & 51.86 & 78.43 \\
        512 & 63 & 38.68 & 64.89 & 91.76 \\
        \bottomrule
    \end{tabular}
\end{table}

This step might be a good area to improve upon so that the algorithm presents a better execution time in bigger instances.

For practical implementations, a possible alternative would be to use faster heuristics instead of the 2-approximation for the SNDP step (see~\cite{Ker2005}). 
Such heuristics could greatly improve the performance of the algorithm, despite loosing theoretical guarantees of the quality of the solutions.

\include{capitulos/ptas_bounded_treewidth}
\chapter{PTAS for Graphs of Bounded Genus}
\label{chapter:ptas_bounded_genus}

As we shall see, the proposed PTAS for \steinercycle\ is inspired by the PTAS for the Steiner Tree Problem due to \cite{Bateni}. Put briefly, the proposed algorithm for \steinercycle\ consists of the following steps:

\begin{enumerate}
    \item The algorithm receives as input a graph \(G_{in}\) with bounded genus, and a set \(\mathcal{D} = \{T_1, \dots, T_k\}\) of terminal pairs.

    \item It begins by creating a spanner subgraph (i.e., a subgraph whose size is bounded by a constant time \(\opt\) and has a valid solution with a cost at most \((1 + \epsilon) \opt\) of \(G_{in}\). This is done with an auxiliary clustering algorithm provided by Theorem~\ref{theoremClustering}.

    \item The edges of the spanner are split into \(k\)-disjoint sets, then the edges in the smallest set are contracted. After the contraction, the algorithm applies \citeauthor{Demaine2010}'s Theorem (\cite{Demaine2010} Theorem~1.1) to convert the spanner subgraph to a bounded treewidth graph.

    \item The algorithm then applies the PTAS for bounded treewidth graphs presented in Chapter~\ref{chapter:ptas_bounded_tree}. This step returns a solution with cost of at most a costant times \(\opt\).

    \item Finally, it constructs the final solution, i.e., the PTAS for bounded genus, by leveraging the solution in bounded treewidth and the set of edges contracted in the previous step.
\end{enumerate}

Similarly to the Chapter~\ref{chapter:ptas_bounded_tree}, we start this chapter by introducing some tooling that will be used in the proof of the PTAS.

\section{Prize Collecting Partition}
\label{section:pc-partition}

This section aims to prove the Prize Collecting Partition Theorem (Theorem~\ref{theoremClustering}) presented below, allowing us to break a Steiner Multicycle instance into simpler, smaller ones. The proof relies upon an algorithm called Prize-Collecting Clustering, or PC-clustering, which will be presented later in this section.

\begin{ftheo}~\label{theoremClustering}
Given an \(\epsilon > 0\), a graph \(G_{in}\), a cost function associated with each edge in \(E(G_{in})\), and a set \(\mathcal{D}\) of terminal pairs, we can compute in polynomial time a set of disjoint components \(\{C_1, \dots, C_k\}\) (i.e., a set of subgraphs of \(G_{in}\)) with an associated partition of the set of terminal pairs \(\{\mathcal{D}_1, \dots, \mathcal{D}_k\}\) with the following properties:
\begin{enumerate}
    \item All sets of terminal pairs are covered, that is, \(\mathcal{D} = \bigcup_{i=1}^k \mathcal{D}_i\); \label{condition_t_clust:1}
    \item All the terminals pairs in \(\mathcal{D}_i\) are covered by the component \(C_i\); \label{condition_t_clust:2}
    \item The sum of the cost of all the components \(C_i\) is no more than \((16/\epsilon + 4) \opt_{\mathcal{D}}(G_{in})\); \label{condition_t_clust:3}
    \item The sum of the costs of the minimum collection of cycles for each set of terminal pairs \(\mathcal{D}_i\) is no more than \(1 + \epsilon\) times the cost of a minimum solution of the SMCP in \(G_{in}\); that is, \(\sum_i \opt_{\mathcal{D}_i} (G_{in}) \leq (1 + \epsilon) \opt_{\mathcal{D}} (G_{in})\). \label{condition_t_clust:4}
\end{enumerate}
\end{ftheo}

The last condition implies that we can solve the problem in each \(C_i\) separately and still guarantee an approximation with a small factor.

We start proving Theorem~\ref{theoremClustering} by applying a 4-approximation algorithm in \(G_{in}\) (for instance, the algorithm for SMCP provided by \cite{Pereira2018TheSM}). Clearly, this construction satisfies the first two properties, since by definition an SMCP solution covers all sets, and for each set \(\mathcal{D}_i\) a component in the solution connects all of its terminal pairs. To guarantee the last two properties, we need to employ the \textbf{PC-clustering} algorithm, which is presented in Section~\ref{subsection:pc_clustering}.

\subsection{Prize-Collecting Clustering}~\label{subsection:pc_clustering}

The PC-Clustering algorithm is used in the proof of the following theorem - which in turn will be used to prove Theorem~\ref{theoremClustering}. 
It is worth mentioning that the Theorem~\ref{theoremClustering_Bateni_3_1} is based on \cite{Bateni} (see Theorem 3.1 (Prize-Collecting Clustering)).

\begin{ftheo} \label{theoremClustering_Bateni_3_1}
Let \(G=(V, E)\) be a graph with a nonnegative edge cost \(c(e)\) for each edge \(e\) and a potential \(\phi_v\) for each vertex \(v \in V\). 
There is a polynomial-time algorithm that computes a subgraph \(Z\) of \(G\) such that:

\begin{enumerate}
    \item the cost of \(Z\) is at most \(4 \sum \limits_{v \in V} \phi_v\), and~\label{condition:1}
    \item \(Z\) is a spanning subgraph of \(G\), and~\label{condition:2}
    \item all vertices in \(Z\) have even degree, and~\label{condition:3}
    \item for any subgraph \(L\) of \(G\), there is a set  of vertices \(Q\) such that: \label{condition:4}
    \begin{enumerate}
        \item \(\sum \limits_{v \in Q} \phi_v\) is at most the cost of \(L\), and \label{condition:4a}
        \item if two vertices \(v_1, v_2 \notin Q\) are connected by \(L\), then they are  in the same component of \(Z\).~\label{condition:4b}
    \end{enumerate}
\end{enumerate}
\end{ftheo}

The statement above might look too abstract at first, but it will get more clear with its proof and application. First let us present a brief overview of the PC-Clustering algorithm, followed by a more complete description.

We start with a graph where vertices have associated potentials (or prizes) and edges have a ``crossing cost''. The idea is to agglomerate the vertices into disjoint clusters progressively. 

Initially, each vertex induces an individual cluster. The total potential of a cluster equals the sum of the potentials of its internal vertices minus the cost of its internal edges. At each step, clusters can use their potential to ``pay'' for crossing edges to merge with other clusters. That way, the algorithm iteratively spends the potential of the graph's vertices until all the potential is spent.

\cite{Bateni} make an analogy to edge painting while presenting the algorithm, where each vertex is associated with a color, and its colors are used to paint edges to connect separated clusters; when an edge is fully painted, it merges two clusters.

To summarize, at the beginning of the algorithm, each vertex \(v\) of \(V(G)\) has a potential \(\phi_v\) that can be spent in clusters that contain \(v\) (so this cluster can connect to other clusters to form new bigger clusters). We keep track of how much ``potential'' the vertex \(v\) spent on a cluster \(S\) via the variable \(\gamma_{S, v}\), where \(S \subseteq V(G): v \in S\). Given a cluster \(C\), we generalize the variable \(\gamma\) as \(\gamma_{C}:= \sum_{v \in C} \gamma_{C, v}\) to account for the total potential spent by any vertex in the cluster \(C\).

The algorithm consists of two stages: \textit{growth} and \textit{pruning}. At the growth stage, we aim to find a subgraph \(F_1\) and a corresponding \(\gamma\). Next, we \textit{prune} (i.e., remove) some of the edges of \(F_1\) to get another subgraph \(F_2\). Finally, we duplicate all edges in \(F_2\) to obtain \(M_2\).

Noticeably, the following inequalities must be respected throughout the whole process.

\begin{align}
&&\sum_{S\subseteq V(G): e \in \delta(S)} \sum_{v \in S} \gamma_{S, v} \leq c(e) && \forall e \in E(G) \label{ineq:1} \\
&&\sum_{S \subseteq V(G): S \ni v} \gamma_{S, v} \leq \phi_v && \forall v \in V \label{ineq:2} \\
&&\gamma_{S, v} \geq 0 && \forall S \subseteq V(G), v \in S \label{ineq:3}
\end{align}

The main goal of the growth stage is to build a spanning subgraph \(F_1\) that partitions the vertices of \(G\). The intuition is to think of \(\phi_v\) as an amount of potential that \(v\) can spend in a cluster \(S\), to which it belongs (i.e., \(v \in S\)), to merge \(S\) with other clusters in \(G\). However, that must be done in a way that the total spent to cross an edge is no more than the cost of that edge (Constraint~\eqref{ineq:1}), and that vertex \(v\) of \(S\) does not use more potential than it has available (Constraint~\eqref{ineq:2}). The Constraint~\eqref{ineq:3} ensures that the value that \(v\) contributes to \(S\) is never negative.

We begin the growth stage with an empty \(\gamma\) (i.e., \(\forall v \in V(G)\) and \(\forall S \subseteq V(G)\) we have \(\gamma_{v, S} = 0\)) and a subgraph \(F_1\) also empty (i.e., has no edges). During this step, we keep track of a set of clusters \(\mathcal{C}\) which partitions the vertices of \(V(G)\); initially, each cluster is composed of a single vertex. We say that a cluster \(C \in \mathcal{C}\) is \textbf{active} if \(\sum_{C' \subseteq C} \sum_{v \in C'} \gamma_{C', v} < \sum_{v \in C} \phi_v\) (i.e. the cluster \(C\) still has vertices with remaining potential). Notice that if a vertex \(v \in C_1 \subset C_2\) spent \(\gamma_{C_1, v}\) to join \(C_1\) with another cluster to form \(C_2\), the potential \(\gamma_{C_1, v}\) cannot be used in \(C_2\). We say that a vertex is active if it still has some spending potential.

During the growth process, we iteratively spend potential from vertices to grow all active clusters by a value \(\eta\) - which may vary between iterations. It is important to note that an equal amount of \(\eta\) is spent on each cluster within the same iteration, and for each cluster, each of its active vertices spends the same amount of potential. Thus, it is possible for the amount spent between vertices of different clusters to be different, depending on the number of active vertices in each cluster.

Let us consider a specific cluster \(C\) and denote by \(\kappa(C)\) as the number of active vertices in \(C\). Each active vertex in \(C\) spends \(\eta / \kappa(C)\) of its available potential into the cluster \(C\). This implies that a total of \(\eta\) will be spent on \(C\) during the iteration.
The value of \(\eta\) is the largest possible value that still adheres to the Constraints~\eqref{ineq:1},~\eqref{ineq:2} and~\eqref{ineq:3} for all active clusters.

After each growth iteration, some constraints might reach equality. If Constraint~\eqref{ineq:2} becomes tight for some vertices, then those vertices become inactive. 
If all vertices inside a cluster become inactive, then we say that the cluster itself becomes inactive. 
In case Constraint~\eqref{ineq:1} becomes tight for some edge \(uv\), then the potential spent by two neighboring clusters was enough to cover the cost of the edge. This implies that there are two clusters \(C_u \ni u\) and \(C_v \ni v\) which we can merge into a new cluster \(C = C_u \cup C_v\), by adding \(uv\) to \(F_1\). We then add the new cluster and remove the two former clusters from \(\mathcal{C}\).

Note that after each growth iteration, at least one of the restrictions will meet equality for some set of vertices and/or edges. Thus, after a polynomial number of iterations, either all restrictions are met in equality, or the potential of all vertices is spent.

The pruning stage iteratively removes some edges from \(F_1\) to form \(F_2\). This process is necessary for proving Lemma~\ref{clustering_connecting_bateni_3_4}. Let \(\mathcal{S}\) be the set containing all sets that were a cluster at some point during the growth stage. It can be easily observed that the clusters in \(\mathcal{S}\) are laminar and the maximal clusters are the clusters of \(\mathcal{C}\). Note that \(F_1[C]\) is connected for each \(C \in \mathcal{S}\).

Let \(\mathcal{B} \subseteq \mathcal{S}\) be the set of all tight clusters at the end of the growth stage, i.e., for each \(S \in \mathcal{B}\) we have \(\sum_{S' \subseteq S} \sum_{v \in S'} \gamma_{S', v} = \sum_{v \in S} \phi_v\). At the beginning of the stage, we initialize \(F_2\) as \(F_1\). While there is a cluster \(S \in \mathcal{B}\) such that \(F_2 \cap \delta(S) = \{e\}\) (notice that if \(F_2 \cap \delta(S)\) has more than one edge we do not prune \(S\)), we remove the edge \(e\) from \(F_2\).

At the end of this process, \(F_2\) does not have edges of \(\delta(C)\), for each \(C \in \mathcal{S}\); in other words, no connected component of \(F_2\) can have two vertices such that one belongs to \(C\) and the other does not. Finally, we duplicate the edges of \(F_2\) to form \(M_2\).

For Algorithm~\ref{algorithm:pc-clustering}, we define \(\mathcal{C}(v)\) as the cluster currently containing the vertex \(v \in V\), that is, \(\mathcal{C}(v):= C\) for any \(v \in C \in \mathcal{C}\).

\begin{algorithm}[H]
\caption{PC-Clustering}
\label{algorithm:pc-clustering}
\begin{algorithmic}[1]

\Require Graph \(G\), and potentials \(\phi_v > 0\).
\Ensure Subgraph \(M_2\).

\State Let \(F_1 \gets (V(G), \emptyset)\).
\State Let \(\gamma_{S, v} \gets 0\) for any \(S \subseteq V : v \in S\).
\State Let \(\mathcal{S} \gets \mathcal{C} \gets \{\{v\}: v \in V(G)\}\).
\While {there is an active vertex} \label{alg:line-while}
    \State Let \(\eta\) be the largest possible value such that simultaneously increasing \(\gamma_C\) by \(\eta\) for all active clusters \(C\) does not violate Constraints \eqref{ineq:1}, \eqref{ineq:2} and \eqref{ineq:3}.
    \State Let \(\gamma_{\mathcal{C}(v), v} \gets \gamma_{\mathcal{C}(v), v} + \frac{\eta}{\kappa(\mathcal{C}(v))}\) for all active vertices \(v\).
    \If {\(\exists e \in E(G)\) that is tight and connects two clusters}
        \State Pick one such edge \(e = \{u, v\}\).
        \State Let \(F_1 \gets F_1 \cup \{e\}\).
        \State Let \(C \gets \mathcal{C}(u) \cup \mathcal{C}(v)\).
        \State Let \(\mathcal{C} \gets \mathcal{C} \cup \{C\} \backslash \{ \mathcal{C}(u), \mathcal{C}(v) \}\).
        \State Let \(\mathcal{S} \gets \mathcal{S} \cup \{C\}\).
    \EndIf
\EndWhile

\State Let \(F_2 \gets F_1\).
\State Let \(\mathcal{B} \gets \{S \in \mathcal{S} : \sum_{S' \subseteq S} \sum_{v \in S'} \gamma_{S', v} = \sum_{v \in S} \phi_v\}\).
\While {\(\exists S \in \mathcal{B}\) such that \(F_2 \cap \delta(S) = \{e\}\) for an edge \(e\)}
    \State Let \(F_2 \gets F_2 \backslash \{e\}\).
\EndWhile
\State Duplicate edges of \(F_2\) to form \(M_2\).
\State Output \(M_2\).

\end{algorithmic}
\end{algorithm}

\begin{flemma}[\cite{Bateni} - Lemma 3.3] \label{clustering_bound_bateni_3_3}
    The cost of \(F_2\) is at most \(2 \sum_{v \in V} \phi_v\).
\end{flemma}
\begin{proof}
The PC-Clustering algorithm has a notion of time. At each discrete step of the algorithm (i.e., at each iteration of line~\ref{alg:line-while}), one of two events happens: a cluster becomes inactive, or an edge becomes tight. We call the ``time'' between two event points \textbf{epoch}. 

Notice that, during each epoch, each cluster is either active or inactive, and each active cluster \(C\) increases its \(\gamma_C\) value, while all other \(\gamma\)'s remain unchanged. At time \(0\), all (singleton) clusters with a strictly positive penalty are active.

We aim at proving that the cost of \(F_2\) is at most \(2 \sum_{S \subseteq V : v \in S} \gamma_{S, v} \leq 2 \sum_{v \in V} \phi_v\), where the inequality follows from Constraint~\eqref{ineq:2}.

Let \(t_j\) be the time at which the \(j^{th}\) event point occurs in the growth step. So the \(j^{th}\) epoch is the time interval between \(t_{j-1}\) to \(t_j\). For each cluster \(C\) let \(\gamma_{C}^{(j)}\) be the amount \(\gamma_{C}:= \sum_{v \in C} \gamma_{C, v}\) that the cluster grew during epoch \(j\), which is  \(t_j - t_{j-1}\) if it was active during this epoch, or zero otherwise. Thus, \(\gamma_C = \sum_j \gamma_C^{(j)}\).

Since each edge \(e\) of \(F_2\) was added at some point by the growth stage when its edge packing Constraint~\eqref{ineq:1} became tight, we can exactly apportion the cost \(c(e)\) amongst the collection of clusters \(\{K: e \in \delta(K)\}\) whose variables ``paid for'' the edge, and can divide this up further by epoch. In other words, \(c(e) = \sum_j \sum_{K : e \in \delta (K)} \gamma_K^{(j)}\). Summing over the epochs yields the desired conclusion.

We will now prove that, for an arbitrary epoch \(j\), the total edge cost of \(F_2\) that is apportioned from epoch \(j\) is at most \(2 \sum_C \gamma_C^{(j)}\). In other words, the total rate at which all active clusters pay for edges of \(F_2\) is at most twice the available potential of the active clusters during the epoch.

Let \(\mathcal{C}_j\) be the set of clusters that existed during epoch \(j\) (either active or inactive). Consider the graph \(F_2\) and then collapse each cluster \(C \in \mathcal{C}_j\) into a supernode. Let \(H\) be the resulting graph. We will continue to refer to each supernode as a ``cluster'' during the rest of the proof. Let us denote the active and inactive clusters in \(\mathcal{C}_j\) by \(\mathcal{C}_{act}\) and \(\mathcal{C}_{dead}\), respectively.

The edges of \(F_2\) that are being partially paid during epoch \(j\) are exactly those edges of \(H\) that are incident to an active cluster -- and the total amount of these edges that are being paid during epoch \(j\) is \((t_j - t_{j - 1}) \sum_{C \in \mathcal{C}_{act}} deg_H(C)\). Since every active cluster grows by exactly \(t_j - t_{j - 1}\) in epoch \(j\), we have:

$$\sum_C \gamma_C^{(j)} \geq \sum_{C \in \mathcal{C}_j} \gamma_C^{(j)} = (t_j - t_{j - 1}) |\mathcal{C}_{act}|$$

Now we show that \(\sum_{C \in \mathcal{C}_{act}} deg_H(C) \leq 2|\mathcal{C}_{act}|\).

Since each cluster induces a connected subtree of \(F_1\) and \(H\) is built from the contraction of edges of \(F_2\), it does not introduce new cycles, which implies that \(H\) is a forest. 
Also, all the leaves in \(H\) must be active, because otherwise the corresponding cluster would have been pruned into another component during the pruning stage.

With this information about \(H\), it is easy to bound \(\sum_{C \in \mathcal{C}_{act}} deg_H(C)\). The total degree in \(H\) is at most \(2 (|\mathcal{C}_{act}| + |\mathcal{C}_{dead}|)\). Noticing that the degree of dead clusters is at least two, we get \(\sum_{C \in \mathcal{C}_{act}} deg_H(C) \leq 2 (|\mathcal{C}_{act}| + |\mathcal{C}_{dead}|) - 2|\mathcal{C}_{dead}| = 2|\mathcal{C}_{act}|\) as desired.

Finally, we obtain the following inequality because $t_j-t_{j-1} \ge 0$:

$$(t_j - t_{j - 1}) \sum_{C \in \mathcal{C}_{act}} \mathrm{deg}_H(C) \leq 2 (t_j - t_{j - 1}) |\mathcal{C}_{act}|$$

Since the left-hand side of the inequality is the total amount being paid to edges of \(F_2\) during epoch \(j\), by summing through all epochs, we get:

$$c(F_2) \leq 2 \sum_j \sum_C \gamma_C^{(j)} = 2 \sum_{S \subseteq V : v \in S} \gamma_{S, v} \leq 2 \sum_{v \in V} \phi_v.$$

\end{proof}

Let \(M_2\) be the graph created by duplicating each edge on \(F_2\). Notice that all vertices in \(M_2\) have even degrees and, besides that, \(c(M_2) \leq 4 \sum_{v \in V} \phi_v\), which is formalized in the following result.

\begin{fcorollary}~\label{corollary_1}
The cost of the graph \(M_2\) is at most \(4 \sum_{v \in V} \phi_v\).
\end{fcorollary}
\begin{proof}
    The result follows immediately from Lemma~\ref{clustering_bound_bateni_3_3}.
\end{proof}

The following lemma due to~\cite{Bateni} gives a sufficient condition for two vertices that end up in the same component of \(F_2\).

\begin{flemma}[\cite{Bateni} - Lemma 3.4] \label{clustering_connecting_bateni_3_4}
    Two vertices \(u\) and \(v\) of \(V(G)\) are connected in \(F_2\) if there exist clusters \(S, S'\) both containing \(u\) and \(v\) such that \(\gamma_{S, v} > 0\) and \(\gamma_{S', u} > 0\).
\end{flemma}
\begin{proof}

    The growth stage connects \(u\) and \(v\) since \(\gamma_{S, v} > 0\) and \(u, v \in S\). Consider the path \(P\) connecting \(u\) and \(v\) in \(F_1\). All the vertices of \(P\) are in \(S\) and \(S'\). For the sake of reaching a contradiction, suppose some edges of \(P\) are pruned. Let \(e\) be the first edge being pruned in \(P\). 

    Thus, there must be a cluster \(C \in \mathcal{B}\) which cuts \(e\) (i.e. \(C\) only contains one endpoint of the edge \(e\)); furthermore, \(\delta(C) \cap E(P) = \{e\}\), since \(e\) is the first edge pruned from \(P\). As \(C\) cuts \(e\), the laminarity of the clusters gives \(C \subset S, S'\).

    In addition, note that \(C\) contains precisely one endpoint of the path \(P\) (as opposed to precisely one endpoint of the edge \(e\)). This holds because if \(C\) contained neither or both endpoints of \(P\), so the cluster \(C\) could not cut \(P\) at exactly one edge.

    Let \(v\) the endpoint of \(P\) in this cluster. We then have \(\sum_{C' \subseteq C} \gamma_{C',v} = \phi_v\) because \(C\) is tight. However, as \(C\) is a \textit{proper} subset of \(S\), this contradicts with \(\gamma_{S, v} > 0\), proving that the supposition is false. The case when \(C\) contains \(u\) is symmetric.
\end{proof}

As mentioned before, \citeauthor{Bateni} associate a color to each vertex so that the vertex ``spends'' its color to fill the edges. Therefore, we say that a cluster \(H\) exhausts a vertex's \(v\) color, when \(\sum_{C \subseteq H} \gamma_{C, v} = \phi_v\). This analogy will be mentioned in the lemmas ahead.

\begin{flemma}[\cite{Bateni} Lemma 3.5]~\label{clustering_bateni_3_5}
    If a subgraph \(H\) of \(G\) connects two vertices \(u, v\) from different components of \(F_2\), then \(H\) exhausts the ``color'' corresponding to at least one of \(u\) and \(v\).
\end{flemma}
\begin{proof}
    Given that \(H\) connects \(u\) and \(v\), suppose that \(H\) exhausts neither \(u\) nor \(v\): there is a set \(S\) containing \(u\) and a set \(S'\) containing \(v\) such that \(\gamma_{S, v} > 0\) and \(\gamma_{S', u} > 0\) and \(E(H) \cap \delta(S) = E(H) \cap \delta(S') = \emptyset\). Since \(H\) connects \(u\) and \(v\), this is only possible if \(u\) and \(v\) are both in \(S\) and \(S'\). By Lemma~\ref{clustering_connecting_bateni_3_4}, this implies that \(F_2\) connects \(u\) and \(v\), a contradiction.
\end{proof}

We can relate the cost of a subgraph to the potential value of the colors it exhausts.

\begin{flemma}[\cite{Bateni} Lemma 3.6] \label{clustering_bateni_3_6}
    Let \(Q\) be the set of colors exhausted by subgraph \(G'\) of \(G\). The cost of \(G'\) is at least \(\sum_{v \in Q} \phi_v\).
\end{flemma}
\begin{proof}
    This is quite intuitive. Recall that vertices have associated ``colors'' which are used to ``color'' the edges of \(G\) during cluster growth. Consider a segment on edges corresponding to cluster \(S\) with color \(v\). At least one edge of \(G'\) passes through the cut \((S, \overline{S})\). Thus, a portion of the cost of \(G'\) can be charged from \(\gamma_{S, v}\). Hence, the total cost of the graph \(G'\) is at least as large as the total amount of colors paid for by \(Q\).
\end{proof}

We  finally have all tools to prove Theorem~\ref{theoremClustering_Bateni_3_1}.

\begin{proof}[Proof of Theorem~\ref{theoremClustering_Bateni_3_1}]
The subgraph \(Z\) is the graph \(M_2\), which corresponds to the forest \(F_2\) with duplicated edges, which already guarantees Condition~\eqref{condition:3}.

Condition~\eqref{condition:1} is given by Corollary~\ref{corollary_1}. 

To assert Condition~\eqref{condition:2}, we must observe that, as a first step of the growth stage of the Algorithm~\ref{algorithm:pc-clustering}, we initialize each vertex as an individual cluster.

For Condition~\eqref{condition:4}, let \(Q\) be the set of vertices exhausted by \(L\). Conditions~\eqref{condition:4a} and~\eqref{condition:4b} are proved by Lemma~\ref{clustering_bateni_3_6} and Lemma~\ref{clustering_bateni_3_5}, respectively.

\end{proof}

\subsection{Proof of PC-Partition Theorem}

The proof of the PC-Partition Theorem (Theorem~\ref{theoremClustering}) induces an algorithm that outputs a set of components of an input graph \(G_{in}\). Such an algorithm is presented in Algorithm~\ref{algorithm:pc-partition}.

\begin{algorithm}[H]
\caption{PC-Partition}
\label{algorithm:pc-partition}
\begin{algorithmic}[1]

\Require Graph \(G_{in}\), a set of costs associated with the edges in \(E{G_{in}}\) and a set of terminal pairs \(\mathcal{D}\)
\Ensure Set of components \(\{C_i\}_{i=1}^k\) such that \(C_i\) covers the terminals in \(\mathcal{D}_i\)
\State Use the algorithm proposed by \cite{Pereira2018TheSM} to find a Steiner Multicycle 4-approximated \(M^\ast = \{C^\ast_1, \dots, C^\ast_k\}\) of \(\mathcal{D}\).
\State Contract each cycle \(C_i^\ast\) in order to create a new graph \(G\).
\State For every \(v \in V(G)\), let \(\phi_v\) be equal to \(\frac{1}{\epsilon}\) times the cost of the cycle \(C_i^\ast\) that was contracted into \(v\), and \(0\) in case there is no such cycle.
\State Let \(M_2\) be the solution produced by the PC-Clustering algorithm on \(G\) and \(\phi_v\).
\State Build \(M\) from \(M_2\) by \emph{uncontracting} all cycles \(C_i^\ast\).
\State Return the set of components \(\{C_i\}_{i=1}^k\), with \(\mathcal{D}_i := \{(s, t) \in \mathcal{D}: s, t \in V(C_i)\}\).

\end{algorithmic}
\end{algorithm}

\begin{proof}[Proof of Theorem~\ref{theoremClustering}]

Let \(G\) be the graph resulting from the contraction of the components of the 4-approximation \(M^\ast\) (as shown in Algorithm~\ref{algorithm:pc-partition}). For each component \(C\) of \(M^\ast\) contracted, resulting in a vertex \(v\), we define a potential \(\phi_v := \frac{1}{\epsilon} \cdot c(C)\) if \(v\) is the result of the contraction of a component of \(M^\ast\) and zero otherwise. 

Let \(Z\) be the subgraph of \(G\) given by Theorem~\ref{theoremClustering_Bateni_3_1}. Let \(Z_{in}\) be the subgraph of \(G_{in}\) obtained from \(Z\) by uncontracting the components of \(M^\ast\) and adding \(M^\ast\) to \(Z_{in}\). Note that \(Z_{in}\) is a spanning subgraph of \(G_{in}\), since \(Z\) is a spanning subgraph of \(G\) (Condition~\eqref{condition:2}).

Considering that \(M^\ast\) is a valid solution, and each vertex in \(Z\) has an even degree (Condition~\eqref{condition:3}), we have that \(Z_{in}\) is a valid solution for the SMCP as well. Let \(\{C_1, \dots, C_k\}\) be the components of \(Z_{in}\), and let \(\mathcal{D}_1, \dots, \mathcal{D}_k\) be the set of terminal pairs covered by those components. It is clear that \(Z_{in}\) satisfies Conditions~\eqref{condition_t_clust:1} and~\eqref{condition_t_clust:2}.

To prove that \(Z_{in}\) satisfies Condition~\eqref{condition_t_clust:3}, note that the cost of \(Z_{in}\) is the cost of \(M^\ast\) (which is at most \(4 \cdot \opt\)) plus the cost of \(Z\) (which, by Theorem~\ref{theoremClustering_Bateni_3_1}, is at most \( 4 \sum_{v \in V} \phi_v = \frac{4}{\epsilon} c(M^\ast) \leq \frac{16}{\epsilon} \opt \) by construction).


\begin{figure}[H]
    \centering
% \tikzset{
%     every node/.style={
%         circle,
%         draw,
%         solid,
%         fill=black!50,
%         inner sep=0pt,
%         minimum width=4pt
%     }
% }
\begin{tikzpicture}[thick,scale=1.8,-,shorten >=2pt]

    \draw (0,0) node {} -- (1,1) [dashed] node {};
    \draw (2, 1.5) node {} -- (3,2) [dashed] node {};
    
    \draw (1,2) node {} -- (2, 1.5) [dashed] node {};

    \draw (3, 0.3) node {} -- (4,0) [dashed] node {};
    \draw (3, 0.3) node {} -- (4,-1) [dashed] node {};
    \draw (0,2) node {} -- (1,2) [dashed] node {};
    \draw (4,1) node {} -- (3, 0.3) [dashed] node {};


    \draw (1,1) node {} -- (2, 1.5) [red] node {};
    \draw (3.03,2) node {} -- (4.03,1) [red] node {};
    \draw (2.03, 1.5) node {} -- (2.03,0) [red] node {};   
    \draw (1,1.04) node {} -- (1,-0.26) [red] node {};
    \draw (1,-0.27) node {} -- (2,0.03) [red] node {};
    \draw (0.05,0) node {} -- (0.05,2) [red] node {};
    \draw (4.03,1) node {} -- (4.03,0) [red] node {};

    \draw  (0,2) node {} -- (1,1) [green] node {};
    \draw (0,0) node {} -- (1,-0.3) [green] node {};
    \draw (1,0.95) node {} -- (2, 1.45) [green] node {};
    \draw (2.97,2) node {} -- (3.98,1) [green] node {};
    \draw (1.97, 1.5) node {} -- (1.97,0) [green] node {};
    \draw (1,-0.33) node {} -- (2,-0.03) [green] node {};
    \draw (0,0) node {} -- (0,2) [green] node {};
    \draw (3.97,1) node {} -- (3.97,0) [green] node {};

    \Vertex[x=1, y=2, color=white]{A}
    \Vertex[x=4, y=1, color=white]{B}
    \Vertex[x=3, y=0.3, color=white]{C}
    \Vertex[x=4, y=-1, color=white]{D}

    \Vertex[x=0, y=2, label=$t_1$, color=white]{t_1}
    \Vertex[label=$t'_1$, color=white]{t_1'}

    \Vertex[x=1, y=1, label=$t_2$, color=white]{t_2}
    \Vertex[x=2, y=0, label=$t'_2$, color=white]{t_2'}

    \Vertex[x=1, y=-0.3, label=$t_3$, color=white]{t_3}
    \Vertex[x=2, y=1.5, label=$t'_3$, color=white]{t_3'}

    \Vertex[x=3, y=2, label=$t_4$, color=white]{t_4}
    \Vertex[x=4, y=0, label=$t'_4$, color=white]{t_4'}

\end{tikzpicture}

\begin{tikzpicture}[thick,scale=1.8,-,shorten >=2pt]
\draw[->, thick] (0,0) -- (0,-1);
\end{tikzpicture}

\begin{tikzpicture}[thick,scale=1.8,-,shorten >=2pt]

    \draw (0,-0.03) node {} -- (1,-0.03) [dashed] node {};
    \draw (0,0) node {} -- (0.5,1) [dashed] node {};
    \draw (1,0) node {} -- (0.5,1) [dashed] node {};

    \draw (1,0) node {} -- (2,1) [dashed] node {};
    % \draw (1,0) node {} -- (2,0) [dashed] node {};
    \draw (2,0) node {} -- (2,1) [dashed] node {};

    \draw (2,0) node {} -- (2.5,-0.5) [dashed] node {};


    \draw (0,0.03) node {} -- (1,0.03) [red] node {};

    \Vertex[x=0.5, y=1, color=white]{A}
    \Vertex[x=2, y=0, color=white]{B}
    \Vertex[x=2.5, y=-0.5, color=white]{C}

    \Vertex[x=0, y=0, label=$c_1$, color=white]{c_1}
    \Vertex[x=1, y=0, label=$c_2$, color=white]{c_2}
    \Vertex[x=2, y=1, label=$c_3$, color=white]{c_3}

\end{tikzpicture}

    \caption{In the top figure, the graph \(G\) is depicted with the approximated solution \(M\) and an optimal solution \(M^\ast\) represented with red and green lines, respectively. The bottom figure illustrates \(G\) contracted by \(M\), where each vertex \(\{c_1, c_2, c_3\}\) corresponds to a component of \(M\). The graph \(L\) consists of the red line and the vertices \(\{c_1, c_2, c_3\}\).}
    \label{fig:theorem_clustering_opt_contracted}
\end{figure}


Let \(Z_{in}^\ast\) be optimal solution for SMCP on \(G_{in}\), and let \(L\) be the subgraph of \(G\) corresponding to \(Z_{in}^\ast\) (obtained by contracting the components of \(M^\ast\)), as shown in Figure~\ref{fig:theorem_clustering_opt_contracted}. 

Let \(Q\) be the set of vertices of \(G\) given by the Condition~\eqref{condition:4} of Theorem~\ref{theoremClustering_Bateni_3_1}. The set \(Q\) might contain (or not) vertices from \(G\) that are a result of the contraction of components of \(M^\ast\). It is important to note that there is no need to compute \(Q\) and \(L\) to execute the algorithm. 

Let \(Q_{in}\) be the subgraph of \(G_{in}\) composed of the components of \(M^\ast\) whose contracted vertices in \(G\) belongs to \(Q\). From Condition~\eqref{condition:4a} of Theorem~\ref{theoremClustering_Bateni_3_1}, we have \(c(Q_{in}) = \epsilon \sum_{v \in Q} \phi_v \leq \epsilon c(L) \leq \epsilon \opt\).

To show that the last condition holds, for every \(\mathcal{D}_i\) (i.e., the set of terminal pairs connected by \(C_i\)), we construct a subgraph \(H_i\) that satisfies the terminal pairs in \(\mathcal{D}_i\). Initially, \(H_i\) is empty. For each terminal pair in \(\mathcal{D}_i\), if the component \(K\) of \(M^\ast\) that satisfies the pair belongs to \(Q_{in}\), then we add \(K\) to \(H_i\). Otherwise, we add the component of \(Z_{in}^\ast\) that satisfies the pair into \(H_i\). It is worth mentioning that the algorithm does not actually compute \(H_i\), because that would require to know \(Z_{in}^\ast\) and \(Q_{in}\).

Observe that each component of \(Q_{in}\) is used in at most one of the \(H_i\)’s: as \(Q_{in}\) is a subgraph of \(Z_{in}\), all the terminal pairs satisfied by a component \(K\) of \(Q_{in}\) belong to the same \(\mathcal{D}_i\).

Furthermore, we claim that each component of \(Z_{in}^\ast\) is used in at most one of the \(H_i\)'s. Suppose that a component \(K\) of \(Z_{in}^\ast\) was used in both \(H_i\) and \(H_j\), i.e., \(K\) satisfies a terminal pair in \(\mathcal{D}_i\) and a terminal pair in \(\mathcal{D}_j\). The components of \(M^\ast\) satisfying these two terminal pairs are not in \(Q_{in}\) (otherwise we would have put these components into \(H_i\) or \(H_j\) instead of \(K\)), thus they correspond to nodes \(v_1, v_2 \notin Q\) in the contracted graph \(G\). Thus \(L\), the contracted version of \(Z_{in}^\ast\), connects two nodes \(v_1, v_2 \notin Q\). In this situation, Condition~\eqref{condition:4b} of Theorem~\ref{theoremClustering_Bateni_3_1} implies that \(v_1\) and \(v_2\) are in the same component of \(Z\) and hence the two terminal pairs are satisfied by the same component of \(Z_{in}\). This contradicts that the two terminal pairs are in two different sets \(\mathcal{D}_i\) and \(\mathcal{D}_j\), since each component \(C_i\) of \(Z_{in}\) attends all terminal pairs in each \(\mathcal{D}_i\).

Since every component of \(Q_{in}\) and every component of \(Z_{in}^\ast\) is used by at most one of the \(H_i\)'s, we have \(\sum_{i=1}^k c(H_i) \leq \opt + c(Q_{in}) \leq (1 + \epsilon) \opt\). Each \(H_i\) was constructed from components of \(Q_{in}\) and \(Z_{in}^\ast\), which establishes the last condition of the theorem.

\end{proof}


\section{Spanner}
\label{section:spanner}

Given a graph \(G\) with a genus of at most \(g\), a set of pairs of terminals \(\mathcal{D}\), and a constant \(\epsilon > 0\), we aim to generate a subgraph \(H\) of \(G\) with the following properties. 

\begin{itemize}
    \item Quasi-optimality Property: There is a collection of cycles in \(H\) that connects all pairs of terminals in \(\mathcal{D}\) and has a maximum cost of \((1 + c \epsilon) \opt_{\mathcal{D}}(G)\) (with \(c\) constant), in particular \(\opt_{\mathcal{D}}(H) \leq (1 + c \epsilon) \opt_{\mathcal{D}}(G)\);
    \item Shortness Property: The total cost of \(H\) is at most \(f(\epsilon, g) \cdot \opt_{\mathcal{D}}(G)\), where \(f(\epsilon, g)\) is a constant dependent on \(\epsilon\) and the genus \(g\) of \(G\).
\end{itemize}

It is worth mentioning that the Quasi-optimatily property is also know in the literature as Spanning property.

\begin{ftheo}\label{theorem:spanner}
Given \(\epsilon > 0\) fixed, a bounded-\textit{genus} graph \(G_{in}\) with costs associted with its edges and a set of pairs of terminals \(\mathcal{D}\), we can calculate in polynomial-time a spanner \(H\) of \(G_{in}\) for Steiner Multicycle with respect to \(\mathcal{D}\).
\end{ftheo}

To build a spanner \(H\), we need to apply the PC-Partition technique introduced in Section~\ref{section:pc-partition}. This technique returns a set of components that will serve as the basis for constructing a new structure called \textit{Mortar Graph}.

Throughout this section, for each component \(C_i\) obtained from Theorem~\ref{theoremClustering}, consider \(T_i\) as a minimum spanning tree of \(C_i\).

\subsection{Mortar Graph}

The \textit{Mortar Graph} was proposed in~\cite{Borradaile2009b} for planar graphs and was later expanded in~\cite{Borradaile2012} for bounded genus graphs.

It is a grid-like subgraph that spans all terminals and has size \(\mathcal{O}(\opt)\). Each face of the Mortar Graph contains a subgraph named \textbf{brick}. The crossing between a brick and the rest of the graph can only be done through a limited set of vertices called \textbf{portals}, which bounds the number of possible solutions.

The Mortar Graph is built for each \(T_i\) obtained from the clustering step (Section~\ref{section:pc-partition}) individually to create a subgraph \(H_i\) of \(G\).

\subsubsection{Cut graph construction}

Given a graph \(G\) of genus \(g\) embedded in a surface of same genus and a connected subgraph \(T\) of \(G\) we define the process of \textbf{cutting} \(G\) with \(T\) by duplicating every edge of \(T\), along with the vertices, and creating a new face in the embedding of the graph inside the duplicated edges of \(T\). This process is illustrated in Figure~\ref{fig:cut_graph_example}.

\begin{figure}[h]
    \centering
    \includegraphics[scale=0.7]{imgs/cut_graph_example.png}
    \caption {Example of cutting process using a subgraph \(T\) to form a face \(H\) inside the graph. (\cite{Borradaile2012}).}
    \label{fig:cut_graph_example}
\end{figure}

% It is worth mentioning that if the subgraph used to do the cutting has cycles, the cutting process may disconnect components of the original graph.

From that, we define a \textbf{cut graph} \(CG\) as a subgraph of \(G\) that when used to cut \(G\) results in a planar graph.

The goal of this section is to find a \textbf{cut graph} \(CG\) of \(G\) that contains all terminals and whose cost is bounded by a constant times \(\opt\). Cutting \(G\) using \(CG\) results in a planar graph with a cycle \(\sigma\) as the boundary, where \(\sigma\) is twice the cost of \(CG\).

Figure~\ref{fig:mortar5} illustrates the process of cutting a graph of genus greater than 0.  Figure~\ref{fig:mortar5}(a) shows a cut graph drawn on a torus, while Figure~\ref{fig:mortar5}(b) shows the result of cutting the surface along the graph: the shaded area is homeomorphic to a disk, and the light area is the additional face of the planarized surface.

\begin{figure}[h]
    \centering
    \includegraphics[scale=0.45]{imgs/mortar5.png}
    \caption {Example of a cut in a graph with \textit{genus} greater than zero. (\cite{Borradaile2012}).}
    \label{fig:mortar5}
\end{figure}


\cite{Borradaile2012} observed that, given a planar graph \(G\) and a spanning tree \(T\) of \(G\), the set of edges \(E(G) - E(T)\) induces a spanning tree \(T^{\star}\) in the dual graph \(G^{\star}\). Furthermore, if \(T\) is a minimum spanning tree of \(G\), then \(T^{\star}\) is a maximum spanning tree of \(G^{\star}\). This result was derived by \citeauthor{Borradaile2012} from the Lemma~1 of~\cite{EPPSTEIN199233}.

The result can be generalized for bounded \textit{genus} graphs: if \(T\) is a minimum spanning tree of \(G\) and \(T^{\star}\) is a maximum spanning tree of \(G^{\star} - E(T)\), then \(T^{\star}\) is a maximum spanning tree of \(G^{\star}\) and the size of the set of remaining edges \(X := E(G) - E(T) - E(T^{\star})\) is \(g\), or the Eulerian \textit{genus} of \(G\), obtained by Euler's formula. \cite{Eppstein} define such a triple \((T, T^{\star}, X)\) as the \textit{tree-cotree decomposition} of \(G\), and shows that a cut graph can be generated from this decomposition.

\begin{figure}[H]
    \centering
\begin{tikzpicture}


    \Vertex[x=0, y=2, color=white, Math,IdAsLabel]{A}
    \Vertex[color=white, Math,IdAsLabel]{B}

    \Vertex[x=1, y=1, color=white, Math,IdAsLabel]{C}
    \Vertex[x=2, y=0, color=white, Math,IdAsLabel]{D}

    \Vertex[x=1, y=-0.3, color=white, Math,IdAsLabel]{E}
    \Vertex[x=2, y=1.5, color=white, Math,IdAsLabel]{F}

    \Vertex[x=3, y=0, color=white, Math,IdAsLabel]{G}
    \Vertex[x=3, y=1.5, color=white, Math,IdAsLabel]{H}

    \Edge[opacity=1.0](A)(B)
    \Edge[opacity=0.3](A)(F)
    \Edge[opacity=1.0](A)(C)
    \Edge[opacity=1.0](C)(F)
    
    \Edge[opacity=0.3](A)(E)
    \Edge[opacity=1.0](C)(E)
    \Edge[opacity=0.3](D)(E)


    \Edge[opacity=0.3](F)(H)
    \Edge[opacity=0.3](H)(G)
    \Edge[opacity=0.3](G)(D)
    \Edge[opacity=0.3](D)(F)
    \Edge[opacity=0.3](F)(G)
    \Edge[color=green](D)(H)

    \Edge[color=red](F)(H)
    \Edge[color=red](F)(D)
    

\end{tikzpicture}
    \caption{Example of \(loop(T, e)\). The tree \(T\) is rooted at the vertex \(F\) and is represented by the black (non-opaque) edges. The edge \(e\) is represented in green and the paths between \(e\) and the vertex \(F\) are displayed in red. The \(loop(T, e)\) is composed of the union of the red and green edges.}
    \label{fig:loop_T_e}
\end{figure}


Let \(T\) be a spanning tree rooted at a vertex \(r\), and let \(e\) be an edge not contained in \(T\). We say that \(loop(T, e)\) is the simple cycle formed by \(e\) and the paths between \(r\) and both ends of \(e\) (exemplified in Figure~\ref{fig:loop_T_e}). Based on the results from \citeauthor{Eppstein},~\cite{Borradaile2012} showed in the following result.

\begin{flemma}[\cite{Borradaile2012}, Lemma 1]
    Given a tree-cotree decomposition \((T, T^{\star}, X)\), \(\{loop(T, e): e \in X\}\), induces a cut graph.
\end{flemma}

The construction of the cut graph for our purposes follows from~\cite{Borradaile2012}, with slight technical changes. We start with \(T_i\), which is a minimum spanning tree of the component \(C_i\) calculated in Algorithm~\ref{algorithm:pc-partition}, and contract it to a vertex \(r\).

We then find a shortest-path tree \(SPT\) rooted at \(r\), uncontract \(r\) back into \(T_i\) and set \(T := T_i \cup SPT\), where \(T\) is a spanning tree of \(G\). With that, we can then find a spanning tree \(T^\ast\) of \(G^\ast - E(T)\) using the results presented above.

Let \(X := E(G) - E(T) - E(T^\ast)\). As the output cut graph we return \(CG := T_i \cup \{loop(T, e): e \in X\}\).

We proceed to cut \(G\) using \(CG\) and duplicate each edge and vertex of \(CG\), thus, generating a cycle \(\sigma\) internal to \(CG\). After this process, let \(G_1\) be the planar graph obtained from \(G\). Finally, we invert the face \(\sigma\) in such a way that \(\sigma\) becomes the new external face of \(G_1\). From construction and Theorem~\ref{theoremClustering}, \(c(\sigma)\) is at most \(c \opt\) where \(c\) is a constant.

\citeauthor{Borradaile2012} refers to the algorithm presented above as \textbf{Planarize algorithm} and formalizes the result with the following lemma.

\begin{flemma}[\cite{Borradaile2012}, Lemma 2]
    The algorithm \textbf{Planarize} returns a cut graph \(CG\) such that cutting \(G\) open along \(CG\) results in a planar graph \(G_p\) with face \(f_\sigma\) whose facial walk \(\sigma\)

    \begin{enumerate}
        \item is a simple cycle,
        \item contains all terminals (some terminals might appear more than once as multiple copies might be created during the cutting process); and
        \item has cost \(c(\sigma) \leq c \opt\) for \(c > 0\) constant.
    \end{enumerate}
\end{flemma}

\subsubsection{Strips}

To continue with the algorithm, we need to present the following definition. Given $\epsilon > 0$ and a graph \(G\), a path \(P\) in \(G\) is \(\epsilon\)-\textit{short} if in each pair of vertices \((x, y) \in P\) the distance between \(x\) and \(y\) in \(P\) is at most \((1 + \epsilon)\) times the distance \(x\) and \(y\) in \(G\), in other words, \(dist_P(x, y) \leq (1 + \epsilon) dist_G(x, y)\).

We proceed to decompose the planar graph \(G_1\) into \textit{strips}. Let \(x\) and \(y\) be two vertices in \(\sigma\). Let \(\sigma[x, y]\) be the path between \(x\) and \(y\) in \(\sigma\) in the counterclockwise direction of the \textit{planar embedding} of \(G_1\) into a sphere. If \(x = y\), by convention \(\sigma[x, y] = \sigma\).

In order to segment \(G_1\) into strips, we find vertices \(x\) and \(y\) in \(\sigma\) such that \(\sigma[x, y]\) is the shortest path of \(\sigma\) which is not \(\epsilon\)-\textit{short} in \(\sigma\). Such a pair of vertices always exists since \(\sigma[x, y]\), with \(x = y\), is not \(\epsilon\)-\textit{short}. Let \(N\) be the shortest path between \(x\) and \(y\) in \(G_1\). Certainly \(c(N) < c(\sigma[x, y])\). We call \textbf{strip} the subgraph of \(G_1\) covered by \(N \cup \sigma[x, y]\). This process is performed recursively on the subgraph of \(G_1\) covered by \(N \cup (\sigma - \sigma[x, y])\). As a result, we have \(G_1\) segmented by strips. Items~(a) and (b) of Figure~\ref{fig:mortar2} illustrate this process.

\begin{flemma}[\cite{klein2006}, inequality (10)] \label{strip_length}
The total cost of the boundary of all strips is at most \((\frac{1}{\epsilon} + 1)\) times the cost of \(\sigma\).
\end{flemma}

\citeauthor{klein2006} also showed that the strip decomposition of a planar graph with \(n\) vertices can be found in time \(\mathcal{O}(n \log n)\).

Given a fixed strip, we denote \(N\) as the north-boundary of the strip and \(S\) as the south-boundary.

With the graph \(G_1\) decomposed into strips, as illustrated in Item~(b) of Figure~\ref{fig:mortar2}, the next step is, for each strip, to calculate the shortest paths, called \textit{columns}. Consider a strip with north and south boundaries, \(N\) and \(S\) respectively. We select vertices \(s_0, s_1, \dots\) in \(S\) as follows. We embed the north strip above the south strip, directing \(S\) and \(N\) from left to right. Let \(s_0\) be the leftmost vertex common to \(S\) and \(N\). By convention, the column \(C_0\) is defined as the shortest path between \(s_0\) and \(N\), in this case, an empty path.

For \(i \geq 1\), find the first vertex in \(S\) (from left to right) such that the cost of the path from \(s_{i-1}\) to \(s_i\) in \(S\) is greater than \(\epsilon\) times the cost of the shortest path from \(s_i\) to \(N\) within the strip, that is, \(dist_S(s_{i-1}, s_i) > \epsilon \cdot dist_{strip}(s_i, N)\). Thus, column \(C_i\) is defined as the shortest path between \(s_i\) and \(N\) in the strip, as illustrated by Item~(c) of Figure~\ref{fig:mortar2}.

\begin{figure}[H]
    \centering
    \includegraphics[scale=0.5]{imgs/mortar2.png}
    \caption{Construction of strips in Mortar Graph (\cite{Borradaile2009b}).}
    \label{fig:mortar2}
\end{figure}

In Figure~\ref{fig:mortar2}, Item~(a) shows the first track is created by a path (dashed) connecting \(x\) and \(y\). The distance between every pair of vertices \(x'\) and \(y'\), between \(x\) and \(y\) on the boundary, is well approximated by the distance on the boundary. We use recursion in the shaded area; Item~(b) shows how a graph is divided into bands (by the dashed lines). A strip is increased in Item~(c). Columns (vertical lines) are taken from the set of shortest paths between the ``low'', or south boundary (dashed line) and the ``up'', or north boundary (solid line).

\begin{flemma}[\cite{klein2006}, Lemma 5.2]
The sum of all column costs in a strip is at most \(\epsilon^{-1} \cdot c(S)\).
\end{flemma}

After having the columns calculated, for each strip, we have a set \(C_0, C_1, \dots, C_s\) of columns of the strip.

For \(\epsilon > 0\), let \(\kappa = \kappa(\epsilon) = 4 \epsilon ^ {-2} (1 + \epsilon ^ {-1})\). We will use this constant on the following definition and for the following lemmas due to \cite{Borradaile2009b}.

% Let \(\mathcal{C}_i = C_i \cup C_{i+\kappa} \cup C_{i+2\kappa} \cup \dots\) for \(i \in \{0, 1, \dots, \kappa - 1\}\).
Let \(\mathcal{C}_i = \bigcup_{j=0}^{\kappa-1} C_{i+j\kappa}\) for \(i \in \{0, 1, \dots, \kappa - 1\}\).

Let \(i^\ast\) be the index that minimizes \(c(\mathcal{C}_i)\). We designate the columns of \(\mathcal{C}_i^\ast\) as \textbf{super-columns}.

\begin{flemma}[\cite{Borradaile2009b}, Lemma 6.5]
The sum of the costs of the super-columns in a strip is at most \(\kappa^{-1}\) times the sum of the costs of the columns in the strip.
\end{flemma}

\begin{flemma}[\cite{Borradaile2009b}, Lemma 6.6] \label{borradaile_2009b_lemma_6_6}
     The sum of the costs of all the super-columns over all strips is at most \(c \epsilon \opt\), where \(\epsilon > 0\) and \(c > 0\) depends on \(\epsilon\).
\end{flemma}

We define as the \textbf{Mortar Graph} \(MG\) of \(G\) the embedded planar subgraph generated by the edges of \(T_i\), the edges of the strips, and the edges of super-columns. This is illustrated in Item~(b) of Figure~\ref{fig:mortar3}.

We note two properties derived from the results presented during the construction of \(MG\). First, let \(Q\) be the set of all terminals of \(G\), by construction, we have that \(Q \subseteq V(MG)\). The second property is presented below.

\begin{flemma} \label{length_mg}
     \(c(MG) \leq k \epsilon c(\sigma)\) with \(k > 0\) constant.
\end{flemma}
\begin{proof}
    The result follows from Theorem~\ref{theoremClustering} and Lemmas~\ref{strip_length} and~\ref{borradaile_2009b_lemma_6_6}.
\end{proof}

\subsubsection{Bricks}

To create a \textbf{brick} (illustrated in  Figure~\ref{fig:mortar3}(c)), for each face \(f\) of \(MG\), we duplicate the border edges and vertices of \(f\). This results in a disconnected induced subgraph of \(G_1\) that is entirely contained within the ``external'' copy of \(f\)'s border (Figure~\ref{fig:mortar4}(c)). The result of this process can be seen in Figure~\ref{fig:mortar4}.

Figure~\ref{fig:mortar4}(a) illustrates the boundary of a face \(f\) of \(MG\) as a cycle of edges (thick edges), possibly with repetition (that is, an edge can occur twice at the border). The light edges are those inside \(f\) in \(G\). Figure~\ref{fig:mortar4}(b) shows an example of brick \(B\) obtained by the process described above. \(B\) has boundary \(\partial B\). Figure~\ref{fig:mortar4}(c) shows a brick \(B\) contained within the ``external'' copy of the border of \(f\).

\begin{figure}[h]
    \centering
    \includegraphics[scale=0.45]{imgs/mortar4.png}
    \caption{Construction of a brick. (\cite{Borradaile2009b}).}
    \label{fig:mortar4}
\end{figure}

The boundary \(\partial B\) of a brick \(B\) is the simple cycle formed by the edges of the boundary. The corresponding face of \(MG\) is called \textbf{mortar boundary} from~\(B\). Each edge of \(MG\) occurs at most in the disjoint union of the boundary of the bricks.

\begin{flemma}[\cite{Borradaile2009b}, Lemma 6.10]
    A mortar boundary \(B\), in counterclockwise order, is the concatenation of four paths \(W\), \(S\), \(E\), \(N\) (west, south, east, north) such that:
    \begin{enumerate}
        \item The set of edges \(B - \partial B\) is nonempty.
        \item Every vertex of \(\mathcal{D} \cap B\) is in \(N\) or in \(S\).
        \item \(N\) is \(0\)-short in \(B\), and every proper subpath of \(S\) is \(\epsilon\)-short in S.
        \item There exist \(k \leq \kappa\) vertices \(s_0, s_1, s_2, \dots, s_k\) ordered from west to east along \(S\) such that, for any vertex \(x\) of \(S[s_i, s_{i+1})\), the distance from \(x\) to \(s_i\) along \(S\) is less than \(\epsilon\) times the distance from \(x\) to \(N\) in \(B\), that is, \(dist_S(x, s_i) < \epsilon dist_B(x, N)\).
    \end{enumerate}
\end{flemma}

\subsubsection{Portals}

The connection between a face \(f\) of \(MG\) and its respective brick \(B\) is performed through a subset of vertices of \(\partial B\) called \textbf{portals}. This connection is made in such a way that, for each portal \(p\) of \(\partial B\), an edge with weight 0 is placed between \(p\) and the vertex of which it was duplicated in \(\partial f\).

For each brick \(B\), we set some vertices of \(\partial B\) as \textbf{portals}. We define \(\theta = \theta(\epsilon) = 18 \cdot \alpha(\epsilon) \cdot \epsilon ^ {-2}\). Where \(\alpha(\epsilon)\) is a constant that will be defined later.

For portal selection, we use the following greedy algorithm. An initial vertex \(v_0\) of \(\partial B\) is selected with uniform probability. Then we define \(v_1\) as the first vertex of \(\partial B\) clockwise such that \(c(\partial B[v_0, v_1]) > c(\partial B) / \theta\). This process is repeated for \(v_i\) in \(\partial B\) until \(v_0 \in V(\partial B (v_{i-1}, v_i))\).

Using the previous construction, we get the following properties given by \cite{Borradaile2009b}.

\begin{flemma}[\cite{Borradaile2009b}, Lemma 7.1] \label{lemma:cobertura}
(Cover property): For any vertex \(x \in \partial B\), there exists a portal \(y\) such that the \(x\)-to-\(y\)-subpath of \(\partial B\) has a maximum cost \(c(\partial B) / \theta\).
\end{flemma}

\begin{flemma}[\cite{Borradaile2009b}, Lemma 7.2]
(Cardinality property): There are at most \(\theta\) portals in \(\partial B\).
\end{flemma}

With that, we have a Mortar Graph \(MG\) with a set of bricks connected with the face boundaries of \(MG\) through portal edges, as illustrated by Item~(d) Figure~\ref{fig:mortar3}. This construction is named by \citeauthor{Borradaile2009b} as \textbf{portal-connected graph}, and is denoted as \(\mathcal{B}^{+}(MG)\).

\begin{figure}[H]
    \centering
    \includegraphics[scale=0.4]{imgs/mortar3.png}
    \caption{Mortar Graph. (\cite{Borradaile2009b}).}
    \label{fig:mortar3}
\end{figure}

Figure~\ref{fig:mortar3} shows in Item~(a) the graph \(G_1\), highlighting the edges of the Mortar Graph. 
Item~(b) shows only the Mortar Graph \(MG\) obtained. 
Item~(c) shows the set of bricks corresponding to \(MG\) (those will be defined ahead). 
Item~(d) illustrates a portal-connected graph \(\mathcal{B}^{+}(MG)\) (will be presented ahead), where the portal edges are gray. 
Item~(e) shows \(\mathcal{\mathcal{B}}^{+}(MG)\) with the bricks contracted into vertices, resulting in \(\mathcal{B}^{\div}(\mathcal{B }^{+}(MG))\).

\subsection{Spanner construction}

Finally, we proceed to prove Theorem~\ref{theorem:spanner}.

We start by building a separate subgraph \(H_i\) for each \(T_i\). Let \(H_i\) be a Mortar Graph created from \(T_i\). Following that, for each brick \(B\) and a selection of its portals \(\Pi' \subseteq \Pi\), we add to \(H_i\) an optimal Steiner Tree that reaches \(\Pi'\) and uses only edges from the interior of \(B\) or its boundary. This can be done in polynomial time in \(\theta\) using the algorithm proposed in \cite{ericksonST}, since all terminals lie on the infinite face of a planar graph.

Note that for fixed \(\epsilon\) and \(g\) there is at most a constant number of portals, hence a constant number of such Steiner Trees, and the cost of each is at most the cost of the boundary of the brick \(B\).

We will leverage an auxiliary result to prove both spanner properties.

\begin{flemma}\label{lemma:borradaile_10_1}
    Let \(G\) be a planar embedded graph and let \(C\) be a subgraph of \(G\) that intersects a \(\epsilon\)-short path \(P\). There is a subpath of \(P\) spanning the vertices of \(C \cap P\) whose total cost is \((1 + \epsilon) c(C)\)
\end{flemma}
\begin{proof}
    Let \(P'\) be the shortest subpath of \(P\) that spans all the vertices of \(C \cap P\). There is a path \(Q\) in \(C\) that connects all vertices of \(C \cap P\). Since \(P\) is \(\epsilon\)-short \(c(P') \leq (1 + \epsilon)c(Q) \leq (1 + \epsilon)c(C)\).
\end{proof}

\begin{flemma} [\cite{Bateni}, Lemma 4.1] \label{bateni_4_1_forest}
    For any forest \(F\) in a brick \(B\), there exists a forest \(F'\) such that
\begin{enumerate}
    \item \(c(F') \leq (1 + \epsilon)c(F)\);
    \item \(F'\) crosses the boundary of \(B\) at most \(\alpha\) times; and
    \item any two vertices on \(N\)- or \(S\)-boundaries of \(B\) connected by \(F\) are also connected by \(F'\)
\end{enumerate}
\end{flemma}

In Corollary~\ref{bateni_4_1_multicycle}, we will generalize the result above to a collection of cycles (instead of a forest), but first, we need to introduce the concept of \textbf{cleaving}.

Put simply, cleaving is the process of breaking a vertex into two new vertices and adding an artificial, zero-cost edge between them. It can be formalized as: given a vertex \(v\) and a bipartition \(A, B\) of the edges incident to \(v\), split \(v\) into two new vertices \(v_A\) and \(v_B\). Then, connect the endpoints of the edges in \(A\), previously connected to \(v\), to \(v_A\), and the edges in \(B\) to \(v_B\). Finally, add a zero-cost edge \(e_{AB}\) between \(v_A\) and \(v_B\). This operation is illustrated in Figure~\ref{fig:cleaving} (a) and (b).

There can be two types of cleaving:
\begin{itemize}
    \item Simplifying Cleaving (Figure~\ref{fig:cleaving} (c) and (d)). Let \(C\) be a clockwise, non-self-crossing (i.e., planar) non-simple cycle that visits a vertex \(v\) twice. Defined a bipartition of the edges incident to \(v\) as follows: given the clockwise embedding of the edges incident to \(v\), let \(A\) start and end with consecutive edges of \(C\) and contain only two edges of \(C\), let \(B\) be the remaining edges. Such a bipartition exists because \(C\) is non-self-crossing.
    \item Lengthening Cleaving (Figure~\ref{fig:cleaving} (e) and (f)). Let \(C\) be a simple cycle, and let \(v\) be a vertex on \(C\) with two edges \(e_A\) and \(e_B\) adjacent to \(v\) embedded strictly inside \(C\), and let \(e'_A\) and \(e'_B\) be consecutive edges of \(C\) adjacent to \(v\) such that the following bipartition is non-crossing concerning the embedding: \(A\), \(B\) is a bipartition of the edges adjacent to \(v\) such that \(e_A, e'_A \in A\) and \(e_B, e'_B \in B\).
\end{itemize}

\begin{figure}[h]
    \centering
    \includegraphics[scale=0.5]{imgs/cleaving.png}
    \caption{Cleavings illustrated (\cite{borradaile_2EC}).}
    \label{fig:cleaving}
\end{figure}

\begin{fcorollary}\label{bateni_4_1_multicycle}
For any collection of cycles \(M\) in a brick \(B\), there is a collection of cycles \(M'\) such that
\begin{enumerate}
    \item \(c(M') \leq (1 + c \epsilon)c(M)\)
    \item \(M'\) crosses the boundary of \(B\) at most an \(\alpha\) (constant) number of times
    \item any two vertices on \(N\)- or \(S\)-boundaries of \(B\) connected by \(M\) are also connected by \(M'\)
\end{enumerate}
\end{fcorollary}
\begin{proof}
    Let \(M^\ast\) be an optimal solution to the SMCP in \(G\) and let \(MG\) be a Mortar Graph built from \(G\). 

    For a given brick \(B\) (with \(W, N, E, S\) boundaries), let \(M\) be the intersection between \(M^\ast\) and \(B\). Note that \(M\) might be composed of cycles and trees (e.g., cycle fragments that connect to themselves outside the brick) and each vertex in \(M \cap \partial B\).

    We begin the construction of \(M_1\) by adding the \(W\) and \(E\) boundaries of \(B\) to \(M\). We know from Lemma~\ref{borradaile_2009b_lemma_6_6} that the cost of the union of \(M^\ast\) with all super-columns (i.e., the west and east boundaries of all bricks) is at most \((1 + 2 c \epsilon) \opt\).

    To streamline the rest of the proof, we perform the simplifying cleaving on non-simple cycles of \(M_1\) until all cycles become simple (also accounting for duplicated edges).
    We can have two situations for each component \(C\) of \(M_1\):
    \begin{enumerate}
        \item \(C\) connects vertices in \(N\) or vertices in \(S\)
        \item \(C\) connects vertices from \(N\) and \(S\)
    \end{enumerate}

    To tackle case (1), we use Lemma~\ref{lemma:borradaile_10_1} to replace the path of \(C\) that connects two vertices in \(N\) with a subpath of \(N\), which connects the same vertices. Since \(N\) is a \(\epsilon\)-path, the Lemma holds, and the increase in cost is limited by a \((1 + \epsilon)\) factor.

    For case (2), we apply the lengthening cleaving to the boundary of brick \(B\). For each vertex \(v\) in \(M_1 \cap \partial B\) that has a degree greater than one, we perform the lengthening cleaving in \(v\). We repeat this process while there are still multiple edges of the solution embedded in a brick that is incident to a shared boundary vertex. In other words, we have that the intersection of \(M_1\) with the interior of brick \(B\) is a subgraph whose joining vertices with \(\partial B\) have degree one.


    The first property follows, since - as mentioned - in case (1), the increase in cost is limited by a \((1 + \epsilon)\) factor, and in case (2) we only add zero-cost edges.
    
    To get the second and third properties, we can directly apply Lemma~\ref{bateni_4_1_forest} (Structural properties of Bricks) on \(M_1\). 
\end{proof}

Considering the final spanner \(H\) to be the union of all \(H_i\)'s, we proceed to prove that \(H\) indeed respects both properties of \textbf{quasi-optimality} and \textbf{shortness}.

\begin{flemma}{{(Shortness property).}}\label{spanner_shortness_property} Given a graph \(G\) and a subgraph \(H\) of \(G\) constructed with the process above, the cost of \(H\) is at most \(f(\epsilon, g) \opt\) for a certain function
\(f(\epsilon, g)\).
\end{flemma}
\begin{proof}
Note that each \(H_i\) is constructed from \(MG_i\) (which by its turn is built from \(T_i\)), a set of portal edges connected to \(MG_i\), and a limited set of Steiner Trees inside the bricks. By Lemma~\ref{length_mg}, \(c(MG_i) \leq f(\epsilon, g) c(T_i)\). For each brick \(B\), we add at most \(2^{\theta}\) trees, each with cost no more than \(c(\partial B)\). Since each edge of \(MG_i\) may appear in at most two bricks, the total cost is bounded by \(2^{\theta + 1} c(MG_i)\). Therefore, \(c(H_i) \leq 2^{\theta + 1} c(MG_i) = f(\epsilon, g) c(T_i)\).

By Theorem~\ref{theoremClustering}, the sum of the cost of all \(T_i\) is no more than \((16/\epsilon + 4) \opt_{\mathcal{D}}(G_{in})\).
This implies that the sum of the cost of all \(H_i\) is no more than \(f(\epsilon, g) (16/\epsilon + 4) \opt_{\mathcal{D}}(G_{in})\).
\end{proof}

\begin{flemma}\label{spanner_quasi_optimality_property}
    Quasi-optimality property.~\(\opt_{\mathcal{D}}(H) \leq (1 + c \epsilon) \opt_{\mathcal{D}}(G_{in})\) with \(c\) and \(\epsilon\) constants.
\end{flemma}
\begin{proof}

    Let \(\{C_i\}_{i=1}^k\) be the set of components outputted by Algorithm~\ref{algorithm:pc-partition}.

    From Theorem~\ref{theoremClustering}, we have that \(\sum_{i}^k \opt_{\mathcal{D}_i} \leq (1 + \epsilon) \opt_{\mathcal{D}}\). So we can focus on proving that \(\opt_{\mathcal{D}_i}(H_i) \leq (1 + c \epsilon) \opt_{\mathcal{D}_i}(G_{in})\).

    Consider each \(H_i\) as formed in the process above, so each \(H_i\) is formed by a Mortar Graph built using \(T_i\) (a minimum spanning tree of \(C_i\) from Theorem~\ref{theoremClustering}), the portal edges and the set of Steiner Trees connecting the portals in each Brick.

    Let \(M^\ast_i\) be an optimal solution for the SMCP considering \(\mathcal{D}_i\), i.e. \(c(M^\ast_i) = \opt_{\mathcal{D}_i}(G_{in})\). We add the set of all super-columns in \(H_i\) to \(M^\ast_i\) to get \(M^1_i\). Recall that, by Lemma~\ref{borradaile_2009b_lemma_6_6}, the cost of these super-columns is at most \(c \epsilon \opt_{\mathcal{D}_i}(G_{in})\).

    Next, we replace the intersection of \(M_i^1\) and each brick with another subgraph having the properties of  Corollary~\ref{bateni_4_1_multicycle}. Let \(M_i^2\) be the new subgraph. The cost of the solution increases to no more than a \(1 + c \epsilon\) factor.

    From  Corollary~\ref{bateni_4_1_multicycle}, we know that \(M_i^2\) crosses each brick at most \(\alpha\) times, so we can ensure that moving these intersection points to the portals adds no more than a constant factor in the cost.

    Consider a brick \(B\) with boundaries \(W, N, E, S\). Connect each intersection point of the brick to its closest portal. Each connection on a brick \(B\) moves by at most \(2 c(\partial B) / \theta\). Therefore, the total movement of each brick is at most \(\alpha 2 c(\partial B) / \theta\), which is no more than \(2 \epsilon^2 c(\partial B) / \gamma(\epsilon, g)\). Hence, the total additional cost for all bricks of \(H_i\) is bounded by \(4 \epsilon^2 c(T_i)\). Let \(M_i^3\) be the resulting subgraph.

    Hence the cost of \(M_i^3\) is at most \(4 \epsilon^2 c(T_i) + c(M_i^2)\). Considering that \(c(M_i^2) \leq (1 + \epsilon) c(M_i^1)\) and \(c(M_i^1) \leq \epsilon^2 c(T_i) + M^\ast_i\), we have that \(c(M_i^3) \leq 4 \epsilon^2 c(T_i) + (1 + \epsilon) c(M^\ast_i)  + \epsilon^2 c(T_i)\)

    Let \(M' = \bigcup_i^k M_i^3\). Accounting that \(\sum_i^k c(T_i) \leq (4/\epsilon + 4) \opt\) (from Theorem~\ref{theoremClustering}), it holds that \[c(M') \leq \sum_i^k \left ( 4 \epsilon^2 c(T_i) + (1 + \epsilon) c(M^\ast_i)  + \epsilon^2 c(T_i) \right).\] 
    Therefore, we can conclude that  \(c(M') \leq (1 + 21 \epsilon + 20 \epsilon^2) \opt = (1 + c'\epsilon) \opt\).
\end{proof}


\section{PTAS for Graphs of Bounded Genus}
\label{section:ptas_bounded_genus}

We can now prove one of the main results of this work, which is a PTAS for the Steiner Multicycle Problem on graphs with bounded genus.

First, we state an auxiliary result from \cite{Demaine2010}.

\begin{ftheo}[\cite{Demaine2010} Theorem 1.1] \label{demaineResult}
    For a fixed genus \(g\), and any integer \(k \geq 2\) and for every graph \(G\) of Euler genus at most \(g\), the edges of \(G\) can be partitioned into \(k\) sets such that contracting the edges in any of the sets results in a graph of treewidth \(\mathcal{O}((g + 1)^2k)\). Furthermore, such a partition can be found in \(\mathcal{O}((g+ 1)^{5/2} n^{3/2} \log{n})\) time.
\end{ftheo}

The proposed PTAS algorithm is presented in Algorithm~\ref{smcp-ptas}.

\begin{algorithm}
\caption{SMCP-PTAS}
\label{smcp-ptas}
\begin{algorithmic}[1]

\Require Graph \(G_{in}\) of bounded genus \(g\), a set of terminal pairs \(\mathcal{D}\), and a constant \(1 \geq \epsilon > 0\).
\Ensure A collection of cycles satisfying \(\mathcal{D}\) whose cost is at most \((1 + c' \epsilon) \opt\)
\State  \(\epsilon' \gets \epsilon / 6\)
\State  \(k \gets \max\left \{ f(g, \epsilon')/{\epsilon'}, \: 2\right \}\); \(f(g, \epsilon')\) as in Lemma~\ref{spanner_shortness_property} \label{alg_smcp_ptas:k}
\State Construct a spanner \(H\) of \(G_{in}\) \label{alg_smcp_ptas:SpannerCall}
\State Use Theorem~\ref{demaineResult} to partition the edges of \(H\) into \(E_1, \dots, E_k\) \label{alg_smcp_ptas:partDemaine}
\State  \(j^\ast \gets \arg\min\limits_{j \in \{1,\ldots,k\}} c(E_j)\)
\State Find a \((1 + k \epsilon)\) collection of cycles (multiset of edges) \(M^\ast\) with respect to \(\mathcal{D}_i\) in \(H/E_{j^\ast}\) using Theorem~\ref{dynamicProgramming}, and with \(k\) constant \label{alg_smcp_ptas:dp}
\State \(M \gets M^\ast \cup E_{j^\ast}\)
\State \Return  \(M\)

\end{algorithmic}
\end{algorithm}

\begin{ftheo}
    The collection of cycles \(M\) produced by Algorithm~\ref{smcp-ptas} on input $(G_{in}, \mathcal{D})$ is a feasible solution for the SMCP and has cost at most \((1 + c' \epsilon) \opt\), with \(c'\) and \(\epsilon\) constants.
\end{ftheo}
\begin{proof}

    In the line~\ref{alg_smcp_ptas:SpannerCall} of Algorithm~\ref{smcp-ptas}, we construct a spanner \(H\) of \(G_{in}\), which respects the properties described in Lemma~\ref{spanner_quasi_optimality_property} and Lemma~\ref{spanner_shortness_property}, that is:

    \begin{itemize}
        \item \(c(H) \leq f(g, \epsilon) \opt\), with \(f(g, \epsilon)\) being a constant that depends on \(\epsilon\) and the genus \(g\) of \(G_{in}\);
        \item \(\opt_{\mathcal{D}}(H) \leq (1 + c \epsilon) \opt_{\mathcal{D}} (G_{in})\).
    \end{itemize}

    Note that Lemma~\ref{spanner_quasi_optimality_property} implies that there is a feasible solution for the SMCP contained in \(H\).

    In line~\ref{alg_smcp_ptas:partDemaine}, we proceed to partition \(H\) using Theorem~\ref{demaineResult}. We choose \(k\) such that \(k = \max( \frac{f(g, \epsilon)}{\epsilon}, 2)\) where \(f(g, \epsilon)\) is the constant from Lemma~\ref{spanner_shortness_property}. 
    
    From Lemma~\ref{demaineResult}, by splitting \(H\) into \(k\) edges sets, we have \(c(E_{j^\ast}) \leq \epsilon \opt\). Moreover, contracting \(E_{j^\ast}\) from \(H\) produces a graph of treewidth \(\mathcal{O}((g +1)^2 k)\).

    Before contracting \(H\) with \(E_{j^\ast}\), we need to do a small treatment of the terminals contained in \(E_{j^\ast}\). We create a new vertex for each terminal in \(E_{j^\ast}\) and connect it to the terminal with a new edge of cost zero. This new vertex will serve as a ``temporary'' terminal in \(H / E_{j^\ast}\). Note that this process does not increase the cost of the final solution, as all added edges have cost zero.

    In line~\ref{alg_smcp_ptas:dp} of Algorithm~\ref{smcp-ptas}, we can find a solution \(M^\ast\) in \(H / E_{j^\ast}\) via Theorem~\ref{dynamicProgramming} and Theorem~\ref{demaineResult}. The cost of \(M^\ast\) is at most \((1 + 2 \kappa \epsilon) \opt\), with \(\kappa = \kappa(\epsilon) = 4 \epsilon ^ {-2} (1 + \epsilon ^ {-1})\).

    Finally, we join the solution \(M^\ast\) with the edges in \(E_{j^\ast}\). To guarantee the feasibility of the union as a solution, we duplicate each edge in \(E_{j^\ast}\). Note that the cost of \(E_{j^\ast}\) accounting for the duplication is at most \(2 \frac{c(H)}{k}\), which equals \(2 \frac{f(g, \epsilon) c(G_{in})}{f(g, \epsilon) / \epsilon} = 2 \epsilon \cdot c(G_{in})\).

    From Theorem~\ref{dynamicProgramming}, we have that \(c(M^\ast) \leq (1 + 2 \kappa \epsilon \opt)\). So \(c(M^\ast \cup E_{j^\ast}) \leq (1 + (2 \kappa + 2) \epsilon) \opt\) holds, and by considering a constant \(c' = 2 \kappa + 2\), we have \(c(M) \leq (1 + c' \epsilon) \opt\).

\end{proof}

The running time of the algorithm, excluding the bounded treewidth PTAS, is bounded by \(\mathcal{O}(n^2 \log n)\). Moreover, the running time of the current procedure for solving bounded treewidth instances is bounded by a polynomial, where \(k\) and \(\epsilon\) appear in the exponent.

\chapter{Conclusion}
\label{chapter:conclusion}

In this work, we first proposed a polynomial-time approximation schema (PTAS) algorithm to solve the Steiner Multicycle Problem (SMCP) on graphs of bounded \textit{genus}. 
This PTAS was built upon ideas from various sources such as \cite{Borradaile2009b}, \cite{Borradaile2012}, and \cite{Bateni}, which we adapted to the Steiner Multicycle Problem, particularly to the restricted version R-SMCP. 
As mentioned in the introduction, we can readily transform instances of SMCP into R-SMCP.
The modifications of the algorithm proposed by~\citeauthor{Bateni} were mainly to extend their definitions and techniques to cycles. 
In particular, it was necessary to adapt most proofs to guarantee that all vertices in the resulting graphs have an even degree.   

We also implemented the 3-approximation algorithm proposed by \cite{smcp_3apx}, which was tested on the same instances previously used by \cite{Pereira2018TheSM}. 
% The algorithm was implemented in C++ using the Graph library \cite{lemon} and the Optimization library \cite{gurobi}.
The experimental results showed that the 3-approximation consistently outperformed the theoretical bounds of solution quality. However, the results were still inferior in both quality and running time when compared to the heuristic proposed by \cite{Pereira2018TheSM}.
One significant bottleneck in the algorithm's performance was the Gomory-Hu trees calculation on the 2-approximation for the SNDP, especially for larger instances.

We believe further improvements are possible, mainly by employing a more robust algorithm in the short-cutting step and by applying a better strategy to verify the flow between terminal pairs in the 2-approximation for the SNDP step.
In particular, for practical implementations, it might be worth considering using a heuristic instead of the 2-approximation in the SNDP step of the algorithm. This heuristic could significantly improve the algorithm's performance, despite loosing theoretical guarantees of the quality of the solutions.

For future works, we propose expanding the algorithm to encompass a more diverse class of graphs, such as H-minor-free graphs. A possible path to achieve this is to use a nearly light subset \((1 + \epsilon)\)-spanner as proposed by \cite{light_spanners_tsp}.

% implementar a aproximação
Another research direction is to implement the proposed PTAS and carry out computational experiments, as \cite{TazariLargeConstants} and \cite{implementationPTASeuclidianTSP} proposed for the Steiner Tree and Euclidean TSP problems, respectively.


% \chapter{Introdução}

% Segundo \cite{horn86robot}, todo triângulo equilátero tem os lados iguais. Já
% segundo \cite{shashua97photometric}, todo quadrado também tem.

% Veja que o pacote \verb|natbib| permite uma série de formas diferentes para
% fazer referências bibliográficas. O comando padrão, \verb|\cite|, realiza a
% citação comum vista no parágrafo anterior. Outros comandos permitem, por
% exemplo, citar somente o autor --- por exemplo, citar o trabalho de
% \citeauthor{samaras99coupled} --- ou colocar automaticamente a citação entre
% parênteses \citep{hougen93estimation, sato99illumination2, sato99illumination1,
% sato01stability}. Os comandos usados foram, respectivamente, \verb|\citeauthor|
% e \verb|\citep|. Veja a documentação do \verb|natbib| na Internet para conhecer
% outros comandos e exemplos de uso.

% Citações aleatórias para fazer com que as referências bibliográficas ocupem
% mais de uma página: \cite{bichsel92simple, dror01statistics, guisser92new}.


% \section{Motivação}

% \dummytxtb\dummytxta

% \subsection{Sub-motivação}


% \dummytxtc\dummytxtb

% \subsection{Mais uma sub-seção}

% \dummytxta\dummytxtc

% \subsubsection{Descendo mais um nível}

% \dummytxtb\dummytxta


% \chapter{Desenvolvimento}

% \dummytxtb\dummytxta\dummytxtc

% \begin{figure}[h]
%     \centering
%     \includegraphics[scale=0.6]{imgs/clustering.png}
%     \caption{Uma figura de exemplo.}
%     \label{fig:exemplo}
% \end{figure}

% \dummytxtb\dummytxta\dummytxtc\dummytxtb

% \begin{table}[t]
%     \caption{Uma tabela de exemplo.}
%     {\centering
%     \begin{tabular}{lcr} \toprule
%     \emph{Left-aligned} & \emph{Centered} & \emph{Right-aligned} \\ \midrule
%     Lorem ipsum & dolor sit & amet \\
%     consectetur adipisicing & elit, sed do eiusmod & tempor \\
%     incididunt ut & labore et dolore & magna aliqua. \\ \bottomrule
%     \end{tabular}\par
%     }
% \end{table}


% Aqui vem a parte da bibliografia: use o comando \ppgccbibliography indicando
% apenas o nome do arquivo .bib (sem a extensão).
\ppgccbibliography{bibfile}


% Este comando encapsula o conjunto de apêndices. A sua função é fazer com que
% a numeração dos apêndices seja feita com letras maiúsculas (A, B, C, etc.) e
% a palavra "Apêndice" anteceda as entradas no Sumário.
% \begin{appendices}

% % Para cada apêndice, um \chapter
% \chapter{Um apêndice}

% \dummytxta
% \dummytxtb
% \dummytxtc
% \dummytxta
% \dummytxtb

% \chapter{Outro apêndice}

% \dummytxta
% \dummytxtb
% \dummytxtc
% \dummytxta
% \dummytxtb

% % Fim dos apêndices (usar apenas depois do último apêndice)
% \end{appendices}


% Este comando encapsula o conjunto de anexos. A sua função é fazer com que a
% numeração dos anexos seja feita com letras maiúsculas (A, B, C, etc.) e a
% palavra "Anexo" anteceda as entradas no Sumário.
% \begin{attachments}

% % Para cada anexo, um \chapter
% \chapter{Um anexo}

% \dummytxta
% \dummytxtb
% \dummytxtc
% \dummytxta
% \dummytxtb

% \chapter{Outro anexo}

% \dummytxta
% \dummytxtb

% % Fim dos anexos (usar apenas depois do último anexo)
% \end{attachments}


\end{document}