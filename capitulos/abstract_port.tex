
No Problema de Multiciclo de Steiner (SMCP), temos um grafo \(G\), uma função de custo $\ell \colon E(G) \to \mathbb{R}_\ge$ e uma coleção de conjuntos de terminais \(\{T_1, \dots , T_k\}\), onde para cada \(a\) em \([k]\) \(T_a\) é um subconjunto de \(V(G)\) e esses conjuntos de terminais são disjuntos. O problema consiste em encontrar um conjunto de custo mínimo de ciclos disjuntos em vértices tal que, para cada \(a\) em \([k]\), todos os vértices de \(T_a\) pertençam a um mesmo ciclo.
Existem aplicações relacionadas a problemas de roteamento que podem ser traduzidas em instâncias do SMCP.
Mais precisamente, o SMCP modela um cenário de transporte de cargas pequenas, onde diversas empresas devem visitar periodicamente os locais de coleta e entrega.
Para reduzir seus custos individuais, essas empresas podem colaborar para criar rotas compartilhadas que visitem vários locais de coleta e entrega, de forma a reduzir os custos totais de transporte.
Claramente, o SMCP é uma generalização do Problema do Caixeiro Viajante e, portanto, não admite um esquema de aproximação em tempo polinomial (PTAS), mesmo no caso métrico.
Um ponto positivo é que a literatura para o SMCP descreve um PTAS para instâncias euclidianas. Nesse trabalho, generalizamos este resultado mostrando um PTAS para SMCP em grafos de largura de árvore limitada e, a partir deste, derivamos um PTAS para grafos de \textit{genus} limitado.
O algoritmo proposto é essencialmente uma adaptação do PTAS para o Problema da Floresta de Steiner em grafos de \textit{genus} limitado.
Além disso, implementamos o algoritmo 3-aproximado para SMCP proposto por \cite{smcp_3apx} em grafos métricos completos e executamos experimentos computacionais.
Os custos das soluções produzidas por este algoritmo são em média \(34\%\) piores do que as soluções ótimas correspondentes, o que é consideravelmente melhor do que a garantia teórica.
Por outro lado, o tempo de execução da implementação é em geral pior do que as outras heurísticas apresentadas na literatura.
\hfill\\

\noindent\textbf{Palavras-chave:} Algoritmos de aproximação, teoria dos grafos, genus limitado, Multiciclos de Steiner, experimentos computacionais