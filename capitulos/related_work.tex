\chapter{Related Work}
\label{chapter:related_work}


The Steiner Multicycle Problem (\(\steinercycle\)) is related to vehicle routing problems, particularly pickup and delivery problems. The literature for this class of problems is vast and considers multiple different restrictions and scenarios (\cite{surveyRouter}).

The \(\steinercycle\) stands between (and generalizes) two big classes of problems that are of much interest to the optimization and theoretical computing communities: the Traveling Salesman Problem (TSP) and the Steiner Tree Problem (STP).

We start this chapter by diving into the literature on the TSP, and some of its variants. We then move to the STP and discuss some of its variants that are more related to this study. 
Our literature review of these problems focuses on approximation algorithms, and more specifically on PTASes.

\section{Travelling Salesman Problem}

One of the most studied vehicle routing optimization problem is the Travelling Salesman Problem or TSP. In this problem, we aim to find a minimal-cost Hamiltonian cycle (i.e., a closed cycle that visits each vertex exactly once) in a given graph.

As mentioned in Chapter~\ref{chapter:introduction}, we can draw a link between the TSP and the \(\steinercycle\), since an instance of the TSP can be converted in polynomial time to an instance of the \(\steinercycle\). That is, the TSP is a particular case of the \(\steinercycle\).

The \textbf{Christofides Algorithm}, proposed by \cite{Christofides2022WorstCaseAO}, is one of the most famous approximation algorithms for the TSP. This algorithm is a \(3/2\)-approximation on metric graphs.

\cite{williamsonApxAlgs} present the following summary of the algorithm. Given a metric graph \(G\), we calculate its minimum spanning tree \(M\). Let \(O\) be the set of odd-degree vertices in \(M\). By the Handshaking Lemma, there are an even number of odd-degree vertices in \(M\). One of the classic results in combinatorial optimization is that given a complete graph (on an even number of vertices), it is possible to calculate a perfect matching of minimum total cost in polynomial time. Considering that, we can calculate it for \(O\). If we add the edges of the perfect matching over \(O\) in \(M\), we create an Eulerian multigraph \(H\) since it is connected and all vertices have even degrees. Finally, we can shortcut the duplicated edges to create a solution of no greater cost corresponding to a Hamiltonian cycle.

Interestingly, more than four decades of research failed to improve upon \citeauthor{Christofides2022WorstCaseAO}' \(3/2\) factor, until \cite{slightlyBetterApxTSP} provided a randomized \((3/2 - \epsilon)\)-approximation for \(\epsilon > 10^{-36}\). Their method follows similar principles to \citeauthor{Christofides2022WorstCaseAO}' algorithm but uses a randomly chosen tree from a carefully selected random distribution in place of the minimum spanning tree.

It is generally known that finding PTASes is difficult for a lot of problems. In particular, \cite{noPTASMetricTSP} showed that there is no PTAS for metric TSP, as well as for the Steiner Tree Problem, Maximum Directed Cut Problem (in which we want to find a bipartition of the vertices of the graph such that the cost of the edges crossing the partition is maximum), and Shortest Superstring Problem (in which we want to find a shortest possible string that contains every string in a given set as substrings), unless \(\poly = \nonpoly\).

\cite{PTASeuclidianTSP} designed a PTAS for \textbf{Euclidean TSP}. They strategy consisted in recursively partitioning the plane (using a randomized variant of the quadtree) such that some \((1 + \epsilon)\)-approximate salesman tour crosses each line of the partition at most \(\mathcal{O}(1/\epsilon)\) times. Such a tour is found by dynamic programming. For each line in the partition, the algorithm first ``guesses'' where the tour crosses this line and the order in which those crossings occur. Then it recurses independently on the two sides of the line. Only \(n \log{n}\) distinct regions are in the partition, where \(n\) is the number of nodes in the plane. Furthermore, the “guess” can be fairly coarse, so the algorithm spends only \(\mathcal{O}(\log{n})^{\mathcal{O}(1/\epsilon)}\) time per region, for a total running time of \(n \cdot (\log{n})^{\mathcal{O}(1/\epsilon)}\). The result still holds even if the nodes are in \(\mathbb{R}^d\), but the running time increases to \(\mathcal{O}(n (\log{n})^{(\mathcal{O}(\sqrt{dc}))^{d-1}})\).

One of the few practical implementations of a PTAS in the literature was done by \cite{implementationPTASeuclidianTSP} for \citeauthor{PTASeuclidianTSP}'s algorithm. Their work describes an implementation of the Euclidean TSP that is based on the essential steps of \citeauthor{PTASeuclidianTSP}'s PTAS with some additional heuristics to improve the running time. In general, their results showed that this PTAS can achieve good performance in practice, although due to challenges during implementation, the authors had to make decisions that caused the loss of theoretical guarantees of the solution's quality. This illustrates how challenging implementing a PTAS can be.

\cite{baker1994} presented a method for obtaining PTASes for a variety of optimization problems in planar graphs, e.g. maximum-weight independent set and minimum-weight vertex cover. This method was generalized by \cite{demaine2005} to derive algorithms for nonlocal problems, such as feedback vertex set and connected dominating set. They were able to derive approximation schemes for subclasses of minor-excluded graphs that involve turning the input graph into a low-treewidth graph. That way, their results apply to graphs that are not planar.

\cite{KleinTSP} improved the results from the work of \cite{basicPTASplanarTSP} by generating an EPTAS for TSP in planar graphs. 
To do so, the authors proposed a framework based on the results from \cite{baker1994} and \cite{demaine2005}, which consists of the following steps.

The first step creates a subgraph \(H\) called \textbf{spanner}.
A spanner \(H\) has two properties of interest: the Quasi-optimality property, which means there is a solution of cost \((1 + \epsilon) \opt\) in \(H\) and the Shortness property, which means the cost of \(H\) is bounded by a constant times \(\opt\). The second step is to find a set of edges of cost at most \(\mathcal{O}(\epsilon \opt)\) contained in \(H\), then proceed to contract those edges. According to~\cite{Demaine2010}, the resulting graph of this process has bounded treewidth. The third step uses a dynamic programming approach on this bounded treewidth graph to obtain an optimal solution. Finally, the solution is the union of the resulting graph from the dynamic programming with the set of edges that was contracted.

\cite{KleinTSP} also showed that this framework could be generalized to obtain PTASes for multiple problems in planar graphs. \cite{Bateni} even commented: ``Most approximation schemes for planar graph problems use (implicitly or explicitly) the fact that the problem is easy to solve on bounded-treewidth graphs''. That fact is the basis for his work on the Steiner Forest Problem (which will be presented later in this chapter).

\cite{EPTASeuclidianTSP} improved on \cite{PTASeuclidianTSP} work by using ``light'' spanners for Euclidean graphs to obtain an EPTAS for Euclidean TSP.~\cite{contraction-decomposition-in-h-minor-free-graphs} proved that any graph excluding a fixed minor can have its edges partitioned into a desired number \(k\) of color classes such that contracting the edges in any one color class results in a graph of treewidth linear in \(k\). This result allowed them to generalize \citeauthor{KleinTSP}'s framework to H-minor-free graphs. Given that, and using the spanners proposed by~\cite{light_spanners_tsp}, \citeauthor{contraction-decomposition-in-h-minor-free-graphs} proposed the first PTAS for TSP in weighted H-minor-free graphs. This result improved upon the early work by~\cite{light_spanners_tsp}, who designed a QPTAS for the same problem.

Since \citeauthor{light_spanners_tsp}'s spanner has weight \(\mathcal{O}(\log{(n)} \; \mathrm{poly}(1 / \epsilon)) \opt\), \citeauthor{contraction-decomposition-in-h-minor-free-graphs}'s PTAS is not efficient.~\cite{eptas-tsp-h-minor-free} improved on the spanner's size resulting in an EPTAS for the problem. The authors mentioned that designing a PTAS for the Subset TSP (also known as Steiner Cycle Problem) in H-minor-free graphs is a central open problem of the field (we will present more on this ahead).

Two related problems to the TSP are the \textbf{Steiner Cycle Problem} and the \textbf{Multiple Steiner Cycle Problem}. These will be presented in the next section.

\section{Steiner-related problems}

We loosely define a ``Steiner-related'' problem as any problem that aims at making some kind of connection between terminals inside one or more sets of vertices. Traditionally, these connections are made with trees or cycles.

During this section, we will present some relevant results in the literature for a few Steiner-related problems, in particular the Steiner Tree Problem, Steiner Cycle Problem, the Steiner Forest Problem, and the Multiple Steiner Traveling Salesman Problem. At the end of the section, we will present some results for the \(\steinercycle\).

\subsection{Steiner Tree Problem}

The Steiner Tree Problem (STP) is one of the most studied problems in the combinatorial optimization literature. The problem consists of finding a tree with minimum cost that connects a subset of the vertices, called terminals, in the graph. It was one of the \(\nonpoly\)-complete problems presented by~\cite{Karp1972}.

In their work,~\cite{Borradaile2009b} drew inspiration from~\cite{KleinTSP} framework to create a PTAS for the Steiner Tree Problem in planar graphs. The authors built a spanner using a subgraph \(MG\), called Mortar Graph, and a set of \textbf{bricks} - subgraphs that are only connected to the rest of the graph via a bounded set of vertices called \textbf{portals}. More details of those concepts are given ahead. This spanner would suffice to be applied to \citeauthor{KleinTSP}'s framework to obtain a PTAS, but it was noted by the authors that such PTAS would yield a time complexity double exponential on \(1 / \epsilon\).

To improve this result, the authors decomposed \(MG\) into a set of subgraphs called \textbf{parcels}. This decomposition was done via a breadth-first search in the dual of \(MG\). As a result, each parcel is an embedded planar subgraph of \(MG\). From this, the authors defined a simple graph called \textbf{parcel graph} in which each vertex represents a parcel in \(MG\).

The constructed parcel graph has some interesting properties, one of which is that the planar dual of each parcel has a spanning tree with depth limited by a constant. That way, they can compute the optimal Steiner Tree for each parcel and take their union to obtain a solution for the original graph. This new approach resulted in a running time singly exponential in \(\mathrm{poly}(1 / \epsilon)\).

To use the Mortar Graph to build the spanner, the authors presented and demonstrated a structural theorem that guarantees that the optimal solution found in the Mortar Graph, and its bricks, is at most \((1 + \epsilon) \opt\).

\subsection{Steiner Cycle Problem}

The \(\steinercycle\) was also studied in a more restricted variant, where we must cover all terminals with a single cycle. This more restricted problem is known in the literature as \textbf{Steiner Cycle Problem}, SCP (it is also known as \textbf{Steiner TSP} or \textbf{Subset TSP}). It is possible to observe that the SCP is equivalent to the TSP in the scenario where all vertices are terminals.

\cite{Cornuejols1985} first introduced the SCP (named Steiner TSP by the authors) while studying the problem in its graphical case and investigating its polyhedron in series–parallel graphs.

\cite{SalazarSteinerCycle} analyzed the polyhedral structure associated with the Steiner Cycle Problem and introduced two lifting strategies to generate inequalities facet defining based on the TSP polytope.

\cite{SteinovaSteinerCycle} presented a 3/2-approximation for the SCP in metric graphs. The authors also showed that there is no constant time approximation for general graphs unless \(\poly = \nonpoly\). This result implies that the SCP, and by consequence the \(\steinercycle\), is NPO-hard in general graphs.

\cite{Arora1998APA} found a \textit{quasipolynomial-time approximation scheme} for the SCP in planar graphs. To find a \((1 + \epsilon)\)-approximated solution for the problem, the proposed algorithm requires time \(\mathcal{O}(n^{\mathrm{poly}(\log n, 1/\epsilon)})\). They made the following conjecture that allows the derivation of a PTAS from this result: ``There exists a function \(f(\cdot)\) such that given \(\epsilon > 0\), a planar graph \(G\) with edge-weights, and a subset \(S\) of vertices, there exists an edge-induced subgraph \(G'\) that \((1 + \epsilon)\)-approximates all distances between nodes in \(S\), and furthermore, \(G'\) has total edge weight at most \(f(\epsilon)\) times the minimum Steiner tree weight for \(S\).''

\cite{klein2006} created a PTAS for SCP in planar graphs, the first one proposed for the problem, by following up from \cite{Arora1998APA}. They confirmed \citeauthor{Arora1998APA}'s conjecture to be true by showing a constructive proof that implies a polynomial-time algorithm for the construction of the subgraph \(G'\), i.e., the \textbf{spanner}.

\cite{klein2014} proposed a sub-exponential parameterized algorithm for the problem with time complexity \((2^{\mathcal{O}(\sqrt{k} \log{k})} + W) \cdot n^{\mathcal{O}(1)}\), where \(n\) is the number of vertices and \(k\) is the numbers of terminals, if \(G\) is a planar graph with weights that are integers no greater than \(W\). The primary strategy to achieve this result consists of two steps: (1) find a locally optimal solution and (2) use it to guide a dynamic program. The proof of correctness of the algorithm depends on the treewidth of a graph obtained by combining an optimal solution with a locally optimal solution.

\cite{LeSubsetTspPTAS_H_MinorFree} proposed a PTAS for the SCP in H-minor-free graphs, for any fixed graph \(H\), expanding on \cite{eptas-tsp-h-minor-free} work. Their main contribution is the concept of a nearly light subset spanner construction based on sparse spanner oracles. They show that spanner oracles with weak sparsity are necessary and sufficient to construct light subset spanners, even for general graphs.

This result is interesting for several reasons. Despite the many advances in meta-algorithms related to PTAS in H-minor-free graphs, a PTAS for SCP on this class of graphs was still unknown. Moreover, there is evidence that a PTAS for this problem is impossible in more general graphs than H-minor-free.


\subsection{Multiple Steiner Traveling Salesman Problem}

The \textbf{Multiple Steiner Traveling Salesman Problem with Order Constraints}, or briefly MSTSPO, is a close variant of the Steiner Multicycle Problem. The difference is that the MSTSPO fixes a set of \(K\) salesman, each having to visit a set of terminals to form the cycles, and considers that each cycle must visit the terminals in a predefined order.

% S. Borne, A. R. Mahjoub, and R. Taktak. A branch-and-cut algorithm for the multiple steiner  TSP with order constraints.
\cite{BORNE2013487} formulated the problem motivated by survivability issues in multilayer telecommunication networks as Multiple Steiner TSP with Order Constraints. Their work proposes an integer linear programming formulation for the problem and investigates the associated polytope. They also present new valid inequalities and discuss their facial aspect. Finally, they devised a Branch-and-cut algorithm and presented preliminary computational results.

% The Multiple Steiner TSP with order constraints: complexity and optimization algorithms
\cite{Gabrel2020} expanded on \citeauthor{BORNE2013487}'s work by proposing a few integer programming formulations and both Branch-and-Cut and Branch-and-Price algorithms to solve the problem. They presented extensive computational results, showing the efficiency of the algorithms.

\subsection{Steiner Forest Problem}

Another \(\steinercycle\) related problem is the Steiner Forest Problem (SFP), where instead of using cycles to connect terminal pairs, we restrict ourselves to using only trees. In a similar way that the \(\steinercycle\) generalizes the SCP, the SFP generalizes the STP. 

\cite{Bateni} proposed a PTAS for the Steiner Forest Problem based on the work of \cite{Borradaile2009b}. To do so, the authors presented a clustering algorithm called \textbf{Prize Collecting Clustering}, which aims at segmenting the original graph in a set of trees \(\{T_1, \dots, T_k\}\) in such a way that the sum of the cost of all trees is at most \((4/\epsilon + 2) \opt\), where \(\opt\) is the cost of an optimal solution for the entire graph, encompassing all terminal sets.

Besides that, considering \(\mathcal{D}_i\) as a set of terminals that must be connected, and \(T_i\) as the tree that connects the terminals in \(\mathcal{D}_i\), the sum of the costs of all the minimal Steiner Forests, connecting only the vertices within each \(\mathcal{D}_i\), totals at most \((1 + \epsilon) \opt\). That can be expressed as \(\sum_i \opt_{\mathcal{D}_i} \leq (1 + \epsilon) \opt\).

In simpler terms, the union of optimal solutions for each terminal set forms an \(1 + \epsilon\)-approximated solution for the input graph. Finally, for each tree \(T_i\) obtained by the clustering, the authors applied the framework proposed by \cite{KleinTSP}, and expanded by \cite{Borradaile2009b, Borradaile2012}.

\subsection{Steiner Multicycle Problem} \label{subsection:steinermulticycle}

% TODO reescrever para ficar diferente da introdução

\cite{LINTZMAYER2020134} proposed the first PTAS for the \(\steinercycle\). The algorithm is a randomized PTAS restricted to the Euclidean plane. In this context, the Euclidean \(\steinercycle\) encompasses a set of terminal pairs distributed across a plane and aim to calculate the line segments that connect the points with the least cost, considering that the same line segment might be crossed more than once.

The algorithm groups the terminal pairs in such a way that pairs that are far away from each other belong to different groups. The authors guarantee that the union of the optimal solution calculated in each group is bounded by a constant times the optimal solution when considering all terminal pairs. Then, for each group of terminal pairs, they create a square called \textbf{root dissection square}, which is a bounding box containing all terminal pairs of the group. This square has a cost of at most a constant times the optimal solution cost considering the terminals pairs in the square.

For each square, they run a process of recursive dissection. In this process, each square is subdivided into four squares of equal size, using vertical and horizontal lines. They select a limited number of points in its border as portals, for each generated square during the partitioning. That way, the solutions generated by the algorithm cross the squares only through the portals. At the end of the process, each square that has not been partitioned is broken into cells, like in a grid. 

For each square \(R\), the authors define a memoization table \(M_R\), indexed by valid configurations of \(R\), in other words, valid subpartitions of the cells and portals of \(R\). Considering a valid configuration of \(R\) called \(\pi\), \(M_R(\pi)\) is the cost of the minimal solution that is \textbf{compatible} with \(\pi\) and \textbf{conforms} to \(R\). A solution is compatible with \(\pi\) when its line segments respect the partition of \(\pi\) and a solution conforms with \(R\) when it is feasible, connects the necessary terminal pairs, only crosses the borders in \(R\) through portals, and makes that crossing a limited number of times.

To write an entry in \(M_R(\pi)\), the authors observe all possible configurations of the squares generated by subdividing \(R\), the ``children'' squares of \(R\), and only consider entries consistent with \(\pi\) from all these configurations. They show that the combined solutions of the four ``children'' squares generate a solution compatible with \(\pi\) and conform to \(R\).

\cite{smcp_3apx} presented a 3-approximation for \(\steinercycle\) on metric and complete graphs. The algorithm was based on the work done by \cite{Christofides2022WorstCaseAO} for the metric TSP and consists of the following three basic steps.
The first step is to perform a 2-approximation for the Survivable Network Design Problem (SNDP), considering a specific set of parameters which will be detailed next. The authors described the SNDP as follows: given a graph \(G\), a weight function \(w: E(G) \rightarrow \mathbb{Q}_+\), and a non-negative integer \(r_{ij}\) for each pair of vertices \(i\), \(j\) with \(i \neq j\), representing a connectivity requirement, the goal is to find a minimum weight subgraph \(G'\) of \(G\) such that, for every pair of vertices \(i, j \in V(G)\) with \(i \neq j\), there are at least \(r_{ij}\) edge-disjoint paths between \(i\) and \(j\) in \(G'\).

They also observed that given an \(\steinercycle\) instance, it is easy to define an equivalent SNDP instance by considering \(r_{ij} = 2\) for any vertices \(i\) and \(j\) that belong to the same terminal set and \(r_{ij} = 0\) otherwise. That way, the optimum value of the SNDP is also a lower bound on the optimum for the Steiner Multicycle Problem: indeed, an optimal solution for the Steiner Multicycle Problem is a feasible solution for the SNDP with the same cost. Those parameters of the SNDP are used on the first step of the algorithm.

Let \(T\) be a set of vertices that has an even number of elements in \(G\). A set \(J\) of edges in \(G\) is a \textbf{T-join} if the set of vertices of \(G\) that are incident to an odd number of edges in \(J\) is exactly \(T\).

Let \(G'\) be the output of the 2-approximation for the SNDP. Consider \(T\) to be the set of odd-degree vertices in \(G'\). Thus, the second step of the algorithm consists of calculating in polynomial time a minimum \(T\)-join \(J\) in \(G'\). Finally, as the third step, we obtain an Eulerian graph \(H\) from \(G'\) by doubling the edges in \(J\) and, by shortcutting an Eulerian tour for each component of \(H\), one obtains a collection \(C\) of cycles in \(G\) that is the output of the algorithm.


\cite{Borradaile2012} expanded the works of \cite{Borradaile2009b} and \cite{KleinTSP} to obtain PTASes in subset connectivity problems in bounded-genus graphs. To do that, the authors proposed two methods. The first method is a generalization of the Mortar Graph framework by \cite{KleinTSP}. The second applies the strategy proposed in \cite{Borradaile2009b}, but using the simpler tree decomposition instead of the parcel decomposition.

% TODO escrever overview aqui

In this work, we intend to expand on the results found by \cite{LINTZMAYER2020134} by proposing a PTAS for the \(\steinercycle\) in graphs with bounded-genus. To that end, we will use various techniques proposed by \cite{Bateni} and \cite{Borradaile2009b} for the Steiner Forest and Steiner Tree problems, respectively, such as the Mortar Graph, the Prize Collecting Clustering, and the spanner.
